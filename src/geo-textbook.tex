
\documentclass{parchment}
\begin{document}

% strona pierwsza
\thispagestyle{empty}
{\noindent\fontsize{18pt}{18pt}\selectfont Biblioteka Aleksandryjska, tom I}

\noindent\makebox[\linewidth]{\rule{\textwidth}{1pt}}

\newpage

% strona druga
\thispagestyle{empty}
\phantom{nothing}
\newpage

% strona trzecia
\thispagestyle{empty}
{\noindent\fontsize{18pt}{18pt}\selectfont Epafrodyt z Ptolemais}

\noindent\makebox[\linewidth]{\rule{\textwidth}{1pt}}

\vspace{10mm}

{\noindent\fontsize{24pt}{24pt}\selectfont \textbf{Geometria}}
\vspace{10mm}

{\noindent\fontsize{14pt}{14pt}\selectfont Wydanie zerowe (eksperymentalne)}

\newpage

% strona czwarta
\thispagestyle{empty}
\begin{figure}[H]
\begin{minipage}[b]{.48\linewidth}
{\noindent Epafrodyt z Eudoksos\\
do napisania\\
do napisania\\
do napisania}
\end{minipage}
\begin{minipage}[b]{.48\linewidth}
{\noindent do napisania\\
do napisania\\
do napisania\\
do napisania}
\end{minipage}
\end{figure}

{\noindent \textbf{Kategorie MSC 2020}\\do napisania} \vspace{5mm}

{\noindent \textbf{Tytuł oryginału}\\do napisania} \vspace{5mm}

{\noindent \textbf{Z greki tłumaczyła}\\do napisania} \vspace{5mm}

{\noindent \textbf{Okładkę zaprojektował}\\do napisania} \vspace{5mm}

{\noindent \textbf{Zredagował}\\do napisania} \vspace{5mm}

{\noindent \textbf{Zredagowała technicznie}\\do napisania} \vspace{5mm}

{\noindent \textbf{Złożyli i połamali}\\do napisania} \vspace{5mm}

{\noindent \textbf{Korekty dokonali}\\do napisania} \vfill

{\noindent Copyleft © 2024 by Antykwariat Czarnoksięski.
Książka, a żeby było śmieszniej także każda jej część, mogą być przedrukowywane oraz w jakikolwiek inny sposób reprodukowane czy powielane mechanicznie, fotooptycznie, zapisywane elektronicznie lub magnetycznie, oraz odczytywane w środkach publicznego przekazu bez pisemnej zgody wydawcy.
}

\vspace{5mm}
{
    \noindent
    Tekst udostępniany na licencji Creative Commons: uznanie autorstwa, użycie niekomercyjne. Przeczytaj więcej na \texttt{https://creativecommons.org/licenses/by-nc/4.0/deed.pl}.
}

\vspace{5mm}

{\noindent Przygotowano w systemie \TeX, wydrukowano na siarczystym papierze.}

% strona piąta
\newpage
\section*{Przedmowa}
Do napisania.

\begin{flushright}
Epafrodyt,\\gdzie, kiedy
\end{flushright}

\tableofcontents
% \pagestyle{fancy} % Enable default headers and footers again
\cleardoublepage % Start the following content on a new page

\chapter{Preludium}

Euklides wyróżnił kilka pojęć pierwotnych (takich jak punkt, który był dla Euklidesa \emph{tym, co nie ma żadnych części}) i pięć aksjomatów, przytoczonych za książką \emph{O Elementach Euklidesa}:

\begin{enumerate}
	\item Zakłada się, że od każdego punktu do każdego punktu można poprowadzić linię prostą.
	\item I że ograniczoną prostą można ciągle przedłużać po prostej.
	\item I że z każdego środka każdym rozwarciem można zakreślić kolo.
	\item I że wszystkie kąty proste są równe między sobą.
	\item I jeżeli prosta padająca na dwie proste tworzy po jednej stronie kąty wewnętrzne, które w sumie są mniejsze od dwóch prostych, to te proste przedłużone nieograniczenie schodzą się po tej stronie, po której kąty te w sumie są mniejsze od dwóch prostych.
\end{enumerate}

Pojęcia pierwotne i aksjomaty Euklidesa nie są jednak idealne.
Dlatego zamiast nich będziemy używać aksjomatów Hilberta podanych około 1899 roku.

\section{Aksjomaty Hilberta}
Wymienimy najpierw wszystkie aksjomaty, a potem przeanalizujemy ich treść.

\begin{itemize}
	\item \textbf{Aksjomaty incydencji}:
\begin{enumerate}
	\item Dla dowolnych dwóch różnych punktów $A, B$ istnieje dokładnie jedna prosta $l$ przechodząca przez nie.
	\item Na dowolnej prostej leżą co najmniej dwa różne punkty.
	\item Istnieją co najmniej trzy różne punkty, nieleżące na jednej prostej.
	\item (Pozostałe aksjomaty incydencji dotyczą przestrzeni trójwymiarowej).
\end{enumerate}
\item \textbf{Aksjomat Playfaire'a}:
\begin{enumerate}
	\item Dla każdego punktu $A$ i każdej prostej $l$, istnieje co najwyżej jedna prosta równoległa do $l$, zawierająca $A$.
\end{enumerate}
\item \textbf{aksjomaty uporządkowania}: \begin{enumerate}
	\item Jeżeli punkt $B$ leży pomiędzy $A$ i $C$, to punkty $A, B, C$ leżą na jednej prostej i punkt $B$ leży pomiędzy $C$ i $A$.
	\item Dla każdych dwóch punktów $A, B$ istnieje punkt $C$, który leży pomiędzy $A$ i $B$.
	\item Dla każdych trzech punktów na prostej, tylko jeden z nich leży pomiędzy pozostałymi dwoma.
	\item (Pasch?) Niech $A, B, C$ będą trzema niewspółliniiowymi punktami, zaś $l$ prostą, która nie przechodzi przez żaden z nich. Jeśli prosta $l$ zawiera punkt $D$ leżący między $A$ i $B$, to musi też zawierać punkt leżący między $A$ i $C$ albo punkt leżący między $B$ i $C$, ale nie obydwa te punkty.
\end{enumerate}
\end{itemize}

Wynika stąd, że dwie różne proste mogą mieć co najwyżej jeden punkt wspólny.

\begin{definition}[równoległość]
	Dwie proste, które pokrywają się albo nie mają żadnych punktów wspólnych, nazywamy równoległymi.
\end{definition}

\begin{definition}[odcinek]
	Niech $A, B$ będą dwoma różnymi punktami.
	Zbiór punktów $A$, $B$ oraz wszystkich punktów leżących pomiędzy $A$ i $B$ nazywamy odcinkiem i oznaczamy $\overline{AB}$.
\end{definition}

% https://en.wikipedia.org/wiki/Playfair%27s_axiom




\section{Przystawanie figur na płaszczyźnie.}
Tekst.

\section{Cechy przystawania trójkątów.}
Tekst.

\section{Własności równoległoboków.}
Tekst.

\section{Problem Fagnano i problem Fermata.}
Tekst.

\section{Kąty w okręgu: wpisane, kąty środkowe i kąty dopisane.}
Tekst.

\section{Twierdzenia o kątach wpisanych, kątach środkowych i kątach dopisanych do okręgu.}
Tekst.

\section{Kątowe warunki na istnienie okręgu przechodzącego przez cztery punkty.}
% https://en.wikipedia.org/wiki/Ptolemy%27s_theorem
% https://en.wikipedia.org/wiki/Casey%27s_theorem
Tekst.

\section{Zastosowanie: okrąg dziewięciu punktów, twierdzenie o prostej Simsona.}
Tekst.

\section{Styczna do okręgu, okrąg wpisany w kąt.}
Tekst.

\section{Okrąg wpisany w trójkąt, okręgi dopisane do trójkąta.}
Tekst.

\section{Warunki istnienia okręgu stycznego do czterech prostych.}
Tekst.



% Neugebauer - rozdział 2

\chapter{Neugebauer 2: twierdzenie Talesa}

It was known to the ancient Babylonians and Egyptians, although its first known proof appears in Euclid's Elements. 
Najstarszy zachowany dowód twierdzenia Talesa zamieszczony jest w VI. księdze Elementów Euklidesa. 
% https://en.wikipedia.org/wiki/Intercept_theorem#Claim_3
\subsection{Twierdzenie Talesa i twierdzenie do niego odwrotne.}

% intercept theorem, also known as Thales's theorem, basic proportionality theorem or side splitter theorem
\begin{theorem}[Talesa]
    Jeśli ramiona kąta płaskiego przetnie się 2 równoległymi prostymi:
    \begin{center}
        \begin{tikzpicture}
            \tkzDefPoint(0, 0.5){O}
            \tkzDefPoint(1.5, 0){A}
            \tkzDefPoint(2, 1){Ap}
            \tkzDefPointBy[homothety=center O ratio 1.618](A) \tkzGetPoint{B}
            \tkzDefLine[parallel=through B](A,Ap) \tkzGetPoint{Bp}
            \tkzInterLL(O,Ap)(B,Bp) \tkzGetPoint{Bpp}
            \tkzDrawPoints[fill=gray,opacity=.9](O,A,B,Ap,Bpp)
            \tkzLabelPoint[above](O){$O$}
            \tkzLabelPoint[below](A){$A$}
            \tkzLabelPoint[below](B){$B$}
            \tkzLabelPoint[above left](Bpp){$B'$}
            \tkzLabelPoint[above left](Ap){$A'$}
            \tkzDrawLine[thick](O,B)
            \tkzDrawLine[thick](O,Bpp)
            \tkzDrawLine[color=blue, thick](A,Ap)
            \tkzDrawLine[color=blue, thick](B,Bpp)
        \end{tikzpicture}
        \end{center}
    to długości odcinków wyznaczonych przez te proste na jednym z ramion kąta są proporcjonalne do długości odpowiednich odcinków na drugim ramieniu kąta, a zatem
    \begin{equation}
        \frac{|OA|}{|OB|} = \frac{|OA'|}{|OB'|} = \frac{|AA'|}{|BB'|}.
    \end{equation}
\end{theorem}

Tradycja przypisuje jego sformułowanie Talesowi z Miletu, chociaż znane było starożytnym Babilończykom i Egipcjanom.
\index[persons]{Tales z Miletu}%
% Pierwszy znany dowód pojawia się w Elementach Euklidesa.
Najstarszy zachowany dowód twierdzenia Talesa zamieszczony jest w VI. księdze Elementów Euklidesa. 
% https://en.wikipedia.org/wiki/Intercept_theorem#Claim_3

\begin{proof}
    Do napisania.
\end{proof}

\begin{proposition}[twierdzenie o odcinku środkowym]
	Jeżeli...
\end{proposition}

\begin{theorem}[Varignona]
	Równoległobok środkowy...
\end{theorem}

%
\subsection{Twierdzenie o sześciu okręgach}

\begin{proposition}[twierdzenie o~sześciu okręgach]
\index{twierdzenie!o sześciu okręgach}%
    Dany są trójkąt $\triangle ABC$ oraz okręgi $K_1$, $K_2$, \ldots, $K_7$ zawarte w~tym trójkącie, wpisane kolejno w~kąty $\angle A$, $\angle B$, $\angle C$, $\angle A$, $\angle B$, $\angle C$, $\angle A$ takie, że okręgi $K_i$ oraz $K_{i+1}$ dla $i = 1, 2, \ldots, 6$ są styczne.
    Wtedy $K_1 = K_7$.
\end{proposition}

(Neugebauer \cite[s. 101]{neugebauer_2018} nazywa to twierdzeniem o~siódmym okręgu).
Tabacznikow, Iwanow \cite{ivanov_tabachnikov_2016} pokazali, że jeśli osłabimy założenia: okręgi nie muszą zawierać się w~trójkącie i~wystarczy, że będą styczne do prostych zawierających boki trójkąta, to nadal ciąg okręgów jest od pewnego miejsca okresowy z okresem równym sześć, ale osiągnięcie tego stanu może wymagać dowolnie wielu kroków.
\index[persons]{Tabacznikow, Siergiej (Табачников, Сергей Львович)}%
\index[persons]{Iwanow, Denis (Иванов, Денис)}%

\begin{proof}
    Evelyn, Money-Coutts, Tyrrell \cite[s. 49–58]{evelyn_money_coutts_tyrrell_1974}.
\index[persons]{Evelyn, Cecil John Alvin}%
\index[persons]{Money-Coutts, Godfrey Burdett}%
\index[persons]{Tyrrell, John Alfred}%
\end{proof}

%

%

\begin{problem}[zadanie Malfattiego]
\index{zadanie Malfattiego}%
	Dany jest trójkąt $\triangle ABC$.
	Skonstruować takie trzy parami styczne okręgi $K_A, K_B, K_C$, że okrąg $K_A$ (odpowiednio: $K_B$, $K_C$) jest wpisany w~kąt $\angle A$ (odpowiednio: $\angle B$, $\angle C$).
\end{problem}

Problem był rozważany na długo przed Malfattim, zajmował się nim matematyk japoński Ajima Naonobu w~XVIII wieku, a~jeszcze wcześniej pojawił się w~rękopisie Gilio di Cecco da Montepulciano z 1384 roku.
\index[persons]{Ajima, Naonobu}%
Malfatti \cite{malfatti_1803} wyprowadził w~1803 roku następujące: niech $r$ będzie promieniem koła wpisanego w~trójkąt, $s$ połową jego obwodu, a~$d_A, d_B, d_C$ odległościami wierzchołków $A, B, C$ od środka koła wpisanego, wtedy promienie kół Malfattiego wyrażają się wzorami
\index[persons]{Malfatti, Gian Francesco}%
\begin{align}
	r_1 & = {\frac {r}{2(s-a)}}(s-r+d_A-d_B-d_C), \\
	r_2 & = {\frac {r}{2(s-b)}}(s-r-d_A+d_B-d_C), \\
	r_3 & = {\frac {r}{2(s-c)}}(s-r-d_A-d_B+d_C).
\end{align}

Prostą konstrukcję okręgów opartą na dwustycznych zawdzięczamy Steinerowi \cite{steiner_1826} w~1826 roku;
\index[persons]{Steiner, Jakob}%
inne rozwiązania podali Lehmus \cite{lehmus_1819}, Catalan \cite{catalan_1846}, Adams \cite{adams_1846}, Derousseau \cite{derousseau_1895}, Pampuch \cite{pampuch_1904}.
% TODO: po poprawie bibliografii, podać tu index persons
(O problemie pisze też Neugebauer \cite[s. 102]{neugebauer_2018}).

Malfatti postawił tak naprawdę inny problem: znalezienia trzech rozłącznych kół zawartych w~trójkącie, których suma pól jest maksymalna i~błędnie założył, że opisane wyżej okręgi stanowią rozwiązanie.
Pomyłkę zauważyli najpierw bez dowodu Lob, Richmond \cite{lob_richmond_1930} w~1930 roku; rygorystyczny dowód, że poprawne rozwiązanie daje algorytm zachłanny podano w~latach sześćdziesiątych tego samego wieku.
% TODO: po poprawie bibliografii, podać tu index persons
% TODO: Lob, H.; Richmond, H. W. (1930), "On the Solutions of Malfatti's Problem for a Triangle", Proceedings of the London Mathematical Society, 2nd ser., 30 (1): 287–304, doi:10.1112/plms/s2-30.1.287.

%

% https://en.wikipedia.org/wiki/Casey%27s_theorem

\color{red}

WIP: Casey w 1866 roku uogólnił twierdzenie Ptolemeusza.

\color{black}

%

% https://en.wikipedia.org/wiki/Taylor_circle

{
    \emph{WIP: Taylor w 1882 roku zauważył, że rzuty spodków wysokości na pozostałe boki leżą na jednym okręgu.}
}

\chapter{Neugebauer 3: współpękowość, współliniowość}

Znamy trzy twierdzenia o współliniowości: ..., ... i twierdzenie o prostej Auberta ...

\begin{proposition}[twierdzenie Salmona]
	Dany jest okrąg oraz trzy jego różne cięciwy $PA$, $PB$, $PC$ takie, że przekrojem okręgów na średnicach $PA$, $PB$ (odpowiednio: $PB$, $PC$ i $PA$, $PC$) są punkty $P$, $M$ (odpowiednio: $P$, $K$ oraz $P$, $L$).
	Wtedy punkty $K$, $L$, $M$ są współliniowe.
\end{proposition}

\begin{proposition}[twierdzenie Menelaosa]
	...
	Wówczas punkty $K, L, M$ są współliniowe wtedy i tylko wtedy, gdy zachodzi
	\begin{equation}
		[AMB] [BKC] [CLA] = -1.
	\end{equation}
\end{proposition}
% https://en.wikipedia.org/wiki/Menelaus%27s_theorem
It is uncertain who actually discovered the theorem; however, the oldest extant exposition appears in Spherics by Menelaus. In this book, the plane version of the theorem is used as a lemma to prove a spherical version of the theorem.

% \begin{proposition}[twierdzenie Carnota???]
	% Neugebauer, strona 108.
% \end{proposition}

% https://en.wikipedia.org/wiki/Newton%E2%80%93Gauss_line#Existence_of_the_Newton%E2%88%92Gauss_line

\begin{proposition}
	Środki trzech przekątnych czworoboku zupełnego leżą na jednej prostej, zwaną prostą Newtona-Gaussa.
\end{proposition}

\begin{proposition}[twierdzenie Desargues'a]
	Neugebauer, strona 109.
	% https://en.wikipedia.org/wiki/Desargues%27s_theorem
\end{proposition}

\begin{proposition}[twierdzenie Pascala]
	Neugebauer, strona 113.
	% https://en.wikipedia.org/wiki/Pascal%27s_theorem
\end{proposition}

\begin{proposition}[twierdzenie Pappusa]
	Neugebauer, strona 114.
	% https://en.wikipedia.org/wiki/Pappus%27s_hexagon_theorem
\end{proposition}

\color{red}

\begin{problem}[zadanie Napoleona]
	Podzielić dany okrąg (bez znanego środka) na cztery łuki równej miary korzystając z cyrkla, ale nie linijki.
\end{problem}

Nie wiadomo, czy Napoleon wymyślił albo rozwiązał przedstawione wyżej zadanie konstrukcyjne.
Rozwiązanie: \cite[s. 116]{neugebauer} z wykorzystaniem okręgów Torricelliego.
\index{okrąg Torricelliego}%

\begin{problem}[zadanie Fermata]
	Dany jest trójkąt $ABC$.
	Znaleźć punkt $F$ taki, by suma $|FA| + |FB| + |FC|$ była możliwie najmniejsza.
\end{problem}

Powyższe zadanie rozwiązał Evangelista Torricelli, który dostał je w formie wyzwania od Fermata.
Rozwiązanie opublikował student Torricelliego, Viviani, w 1659 roku.
% TODO: Johnson, R. A. Modern Geometry: An Elementary Treatise on the Geometry of the Triangle and the Circle. Boston, MA: Houghton Mifflin, pp. 221-222, 1929.

% TODO: rozwiązanie https://en.wikipedia.org/wiki/Napoleon%27s_problem

\color{black}

% CARNOT
Uogólnieniem twierdzenia o współpękowości symetralnych boków trójkąta jest:

\begin{proposition}[twierdzenie Carnota]
	... wtedy i tylko wtedy, gdy
	\begin{equation}
		|AM|^2 + |BK|^2 + |CL|^2 = |AL|^2 + |CK|^2 + |BM|^2.
	\end{equation}
\end{proposition}
% CARNOT

% CEVA
Ważnym kryterium współpękowości trzech czewian jest:

\begin{proposition}[twierdzenie Cevy (1678)]
	Dany jest trójkąt $ABC$ i trzy różne od wierzchołków punkty $K \in BC$, $L \in CA$, $M \in AB$.
	Wówczas czewiany $AK$, $BL$, $CM$ są współpękowe wtedy i tylko wtedy, gdy
	\begin{equation}
		[AMB] [BKC] [CLA] = 1.
	\end{equation}
\end{proposition}

\textbf{Do zrobienia: czewiany Gergonne'a}.
\index{czewiany Gergonne'a}

\textbf{Do zrobienia: czewiany Nagela}.
\index{czewiany Nagela}
% CEVA

% BRIANCHON
Twierdzenie Brianchona, Newtona.
% BRIANCHON

% VAN AUBEL
Twierdzenie van Aubela.
Wzór Routha, równosć Gergonne'a.
% VAN AUBEL

% STEINER
Twierdzenie Steinera.
% STEINER

% LEMOINE
Czewiany, symediany, punkt Lemoine'a.
% LEMOINE

Inwersja, Feuerbach.

Okręgi Apoloniusza, dwustosunek.

\chapter{Neugebauer 4: przekształcenia płaszczyzny}
\begin{enumerate}
	\item Grupa izometrii na płaszczyźnie.
	\item Konstrukcja obrazu punktu, okręgu, prostej przy translacji, obrocie i symetrii osiowej.
	\item Złożenie dwóch i złożenie trzech symetrii osiowych. 
	\item Twierdzenia o składaniu izometrii. Klasyfikacja izometrii na płaszczyźnie. 
	\item Izometrie parzyste i izometrie nieparzyste. Twierdzenie o redukcji.
	\item Twierdzenie Napoleona: środki ciężkości trójkątów równobocznych zbudowanych na bokach dowolnego trójkąta są wierzchołkami trójkąta równobocznego.
\end{enumerate}

Izometrie, punkty stałe.
Translacje, symetrie osiowe, symetrie środkowe, obroty.
Twierdzenie Chasles'a: każda izometria płaszczyzny jest złożeniem co najwyżej trzech symetrii osiowych.
Symetria osiowa z poślizgiem.
Słowo Banacha.
Klasyfikacja podobieństw.
Okrąg siedmiu punktów. % https://mathworld.wolfram.com/BrocardCircle.html ?
Przekształcenia afiniczne i rzutowe.

% https://www.cut-the-knot.org/Curriculum/Geometry/HeronsProblem.shtml
% This one is a basic optimization problem. It's quite famous, being discussed in Heron's Catoptrica (On Mirrors from the Greek word Katoptron Catoptron = Mirror) that, in all likelihood, saw the light of day some 2000 years ago.


\chapter{Gryzmoły}




\section{Wyniki w Elementach Euklidesa}
\section{Pons asinorum}
The theorem appears as Proposition 5 of Book 1 in Euclid's Elements. Its converse is also true: if two angles of a triangle are equal, then the sides opposite them are also equal. 

\textbf{Do zrobienia: wiki}
% https://en.wikipedia.org/wiki/Pons_asinorum

Twierdzenie o dwusiecznej % https://en.wikipedia.org/wiki/Angle_bisector_theorem
The angle bisector theorem appears as Proposition 3 of Book VI in Euclid's Elements. 

The exterior angle theorem is Proposition 1.16 in Euclid's Elements, which states that the measure of an exterior angle of a triangle is greater than either of the measures of the remote interior angles. This is a fundamental result in absolute geometry because its proof does not depend upon the parallel postulate. % https://en.wikipedia.org/wiki/Exterior_angle_theorem

Konstrukcja pierwiastka z iloczynu:
The theorem is usually attributed to Euclid (ca. 360–280 BC), who stated it as a corollary to proposition 8 in book VI of his Elements. In proposition 14 of book II Euclid gives a method for squaring a rectangle, which essentially matches the method given here. Euclid however provides a different slightly more complicated proof for the correctness of the construction rather than relying on the geometric mean theorem.
% https://en.wikipedia.org/wiki/Geometric_mean_theorem


Hinge theorem % https://en.wikipedia.org/wiki/Hinge_theorem

twierdzenia geometrii koła:
- % https://en.wikipedia.org/wiki/Thales%27s_theorem
- The inscribed angle theorem states that an angle $\theta$ inscribed in a circle is half of the central angle $2\theta$ that subtends the same arc on the circle. 

% https://en.wikipedia.org/wiki/Intercept_theorem

% https://en.wikipedia.org/wiki/Inscribed_angle#Theorem

% https://en.wikipedia.org/wiki/Intersecting_chords_theorem
% https://en.wikipedia.org/wiki/Intersecting_secants_theorem
% https://en.wikipedia.org/wiki/Tangent%E2%80%93secant_theorem
% https://en.wikipedia.org/wiki/Power_of_a_point#Theorems
twierdzenie o siecznych

prawo kosinusów
% https://en.wikipedia.org/wiki/Law_of_cosines

Pitagorasa % https://en.wikipedia.org/wiki/Pythagorean_theorem
% https://en.wikipedia.org/wiki/Spiral_of_Theodorus

gnomon % https://en.wikipedia.org/wiki/Theorem_of_the_gnomon

\section{Inne gryzmoły}




Potęga punktu względem okręgu?

Twierdzenie o odcinku środkowym: odcinek łączący środki dwóch boków trójkąta jest równoległy do podstawy i ma połowę jej długości.

Symetria osiowa.
Symetralna: przecinają się w jednym punkcie.

Okrąg.
Styczne, sieczne.
Twierdzenia geometrii koła o miarach kątów.
Twierdzenie Apoloniusza.

Czworokąt cykliczny.
Twierdzenie o prostej Wallace'a-Simsona.

Ortocentrum i trójkąt ortyczny.

Twierdzenie Miquela.

Twierdzenie Pitagorasa.

Twierdzenie Varignona.
Podobieństwo, skala.

Twierdzenie Ptolemeusza.
Twierdzenie Carnota.

Sieczne i styczne.
Potęga punktu względem okręgu.

Twierdzenie o prostej Auberta.

Twierdzenie o dwusiecznej.
Twierdzenie o okręgu Apoloniusza.

Dwustosunek.
Pęki okręgów.

Twierdzenie Ponceleta.

Prosta/twierdzenie Eulera.

Twierdzenie Morleya
Okrąg dziewięciu punktów

Trygonometria - sinusów, cosinusów, Stewarta.
Wzór Herona.
Wzór Brahmagupty

Twierdzenie Urquharta

Punkt i kąt Crelle'a-Brocarda

\section{Geometria I UW (sylabus)}
\begin{enumerate}
	\item tu coś było
	\item -- \begin{enumerate}
		\item Stosunek podziału wektora.
		\item Twierdzenie Talesa, twierdzenie odwrotne i jego zastosowania.
		\item Pole.
		\item Pola wybranych figur, twierdzenie Pitagorasa.
		\item Pole zorientowane.
		\item Twierdzenie Newtona: środek okręgu wpisanego w czworokąt i środki przekątnych tego czworokąta są współliniowe.
		\item Twierdzenie Gaussa: środki przekątnych czworokąta zupełnego są współliniowe.
		\item Definicja jednokładności, podobieństwo figur.
		\item Cechy podobieństwa trójkątów.
		\item Stosunek pól figur podobnych.
		\item Iloczynowe warunki istnienia okręgu przechodzącego przez cztery punkty.
		\item Pojęcie potęgi punktu względem okręgu.
		\item Twierdzenie Ptolemeusza.
	\end{enumerate}
	\item -- \begin{enumerate}
		\item Wielkości miarowe w trójkącie: wzór Herona, wzory na promienie okręgów wpisanych, dopisanych.
		\item Twierdzenie o dwusiecznej i okrąg Apoloniusza.
		\item Twierdzenie Cevy (wraz z trygonometryczną wersją), przykłady punktów szczególnych trójkąta: punkt Nagela, punkt Gergonne'a, punkt Lemoine'a.
		\item Punkty izogonalnie sprzężone w trójkącie.
		\item Twierdzenie Menelausa.
	\end{enumerate}
	\item -- \begin{enumerate}
		\item Jednokładność.
		\item Konstrukcja obrazu jednokładnego punktu, okręgu, prostej.
		\item Środek jednokładności dwóch trójkątów.
		\item Środki jednokładności dwóch okręgów.
		\item Prosta Eulera w trójkącie (środek okręgu opisanego, środek ciężkości, ortocentrum).
		\item Zastosowanie: Twierdzenie Pascala.
		\item Twierdzenie Kirkmana: jeśli część wspólna dwóch trójkątów wpisanych w okrąg jest sześciokątem wypukłym, to główne przekątne tego sześciokąta przecinają się w jednym punkcie.
		\item Grupa dylatacji na płaszczyźnie.
		\item Twierdzenia o składaniu jednokładności i przesunięć, twierdzenie o środkach jednokładności trzech okręgów.
	\end{enumerate}
	\item -- 
	\item -- \begin{enumerate}
		\item Grupa podobieństw płaszczyzny.
		\item Podobieństwa spiralne i odbicia dylatacyjne.
		\item Klasyfikacja podobieństw płaszczyzny.
	\end{enumerate}
\end{enumerate}

\section{Geometria II UW}
1. Potęga punktu względem okręgu, oś potęgowa dwóch okręgów, środek potęgowy trzech okręgów, twierdzenie Brianchona, konstrukcja stycznej do okręgu samą linijką, okręgi współpękowe, twierdzenie Gaussa-Bodenmillera, twierdzenie o motylku, formuła Eulera na odległość między środkami okręgu opisanego i wpisanego (dla trójkąta), twierdzenie Ponceleta dla trójkąta.

2. Obrazy inwersyjne okręgów i prostych, konforemność inwersji, okręgi stałe inwersji, okręgi prostopadłe, zmiana odległości przy inwersji, twierdzenie Ptolemeusza, zmiana promienia okręgu przy inwersji, łańcuchy Steinera, formuła Kartezjusza, formuła Fussa dla czworokątów, twierdzenie Feuerbacha.

3. Ogniska elipsy i hiperboli, ognisko, kierownica i mimośród stożkowych, asymptoty hiperboli, konstrukcja stycznej do stożkowej, rzuty ustalonego ogniska na styczne, własności izogonalne stożkowych, równania kanoniczne stożkowych, elipsa jako przekrój walca. Ognisko, kierownica i mimośród stożkowej na przekroju stożka. Przekroje stożków ze sferami wpisanymi. Równanie ogólne stożkowej w układzie współrzędnych, duży i mały wyznacznik. Równania stożkowych we współrzędnych biegunowych.

4. Grupa przekształceń afinicznych od strony geometrycznej: powinowactwa osiowe, rozkład przekształcenia afinicznego na podobieństwo i powinowactwo osiowe, kierunki główne przekształcenia afinicznego, niezmienniczość stosunku pól przy przekształceniu afinicznym, obraz okręgu przy przekształceniu afinicznym
Literatura: 	

1. E. H. Askwith, D.D. A Course of Pure Geometry, Cambridge 1917.

2. H. Fukagawa, D. Pedoe, Japanese temple geometry problems. Sangaku, Charles Babbage Research Centre, Winnipeg 1989.

3. R. A. Johnson, Advanced Euclidean geometry: An elementary treatise on the geometry of the triangle and the circle, Dover Publications, Inc., New York 1960.

4. W. Pompe, Geometria kół, Wydawnictwo Szkolne OMEGA, Kraków 2019.

5. V. Prasolov, Zadaczi po planimietrii. Tom I-II (ros.), Nauka, Moskwa 1991

\section{Geometria III}
Geometria rzutowa: ujęcie od strony geometrycznej. Płaszczyzna rzutowa (rzeczywista), przekształcenia rzutowe prostych, pęków, stożkowych, pęków stycznych do stożkowych.
{\color{gray}
Twierdzenia Desarguesa, Pappusa, Pascala, Brianchona.
}
Dualność: biegun i biegunowa względem okręgu i stożkowych. Sprzężenie biegunowe. Inwolucje rzutowe, twierdzenia inwolucyjne. Pęki okręgów i stożkowych jako generatory inwolucji. Twierdzenie Ponceleta. Stożkowe w ujęciu rzutowym, twierdzenia Steinera i Braikenridge'a-Maclaurina. Rzutowe określenie ogniska i kierownicy stożkowych. Punkty urojone przecięcia prostej ze stożkową w ujęciu czysto geometrycznym.

\section{Nie wiem skąd}
Aksjomaty. Kąty naprzemianległe i odpowiadające.
Przystawanie trójkątów.

Łamane i wielokąty.
Równoległobok.
Równoważność wektorów.
Symetria osiowa.
Symetralna.
Styczna do okręgu.
Kąty środkowe i wpisane.
Cykliczność. Prosta Wallace'a.
Ortocentrum i trójkąt ortyczny.
Twierdzenie Miquela.
Dwusieczna. Okrąg wpisany i dopisane.
Twierdzenie Pitagorasa.
Twierdzenie Talesa.
Podobieństwo.
Twierdzenie Ptolemeusza.
Twierdzenie Carnota.
Potęga punktu względem okręgu.
Okrąg Apoloniusza.
Pęki okręgów.
Twierdzenie Eulera.
Twierdzenie Morleya.
Trygonometria. Wzór Herona.
Twierdzenie Urquharta.
Kąt Crelle'a-Brocarda.
Twierdzenie o siódmym okręgu.
Współliniowość.
Współpękowość.
Ceva i Menelaos.
Twierdzenie Ponceleta.
Jednokładność.
Inwersja.
Dwustosunek.

% https://en.wikipedia.org/wiki/Problem_of_Apollonius
% https://en.wikipedia.org/wiki/Poncelet%E2%80%93Steiner_theorem
% https://en.wikipedia.org/wiki/Compass_equivalence_theorem
% https://en.wikipedia.org/wiki/Angle_trisection
% https://en.wikipedia.org/wiki/Mohr%E2%80%93Mascheroni_theorem

% burdel

% https://en.wikipedia.org/wiki/Pasch%27s_theorem

% https://en.wikipedia.org/wiki/Apollonius%27s_theorem jest specjalnym przypadkiem Stewarta

% https://en.wikipedia.org/wiki/Aristarchus%27s_inequality

% https://en.wikipedia.org/wiki/Ptolemy%27s_inequality

% https://en.wikipedia.org/wiki/Diophantus_II.VIII

% https://en.wikipedia.org/wiki/Pappus%27s_area_theorem

% https://en.wikipedia.org/wiki/Law_of_sines

% https://en.wikipedia.org/wiki/Heron%27s_formula
% https://en.wikipedia.org/wiki/Brahmagupta%27s_formula

% https://en.wikipedia.org/wiki/Problem_of_Apollonius

% https://en.wikipedia.org/wiki/Pitot_theorem

% https://en.wikipedia.org/wiki/Brahmagupta_theorem

% https://en.wikipedia.org/wiki/Japanese_theorem_for_cyclic_quadrilaterals
% https://en.wikipedia.org/wiki/Japanese_theorem_for_cyclic_polygons

% https://en.wikipedia.org/wiki/Kosnita%27s_theorem

% https://en.wikipedia.org/wiki/Musselman%27s_theorem

% https://en.wikipedia.org/wiki/Harcourt%27s_theorem

% https://en.wikipedia.org/wiki/Feuerbach_point

% https://en.wikipedia.org/wiki/Euler%27s_theorem_in_geometry

% https://en.wikipedia.org/wiki/Equal_incircles_theorem

% https://en.wikipedia.org/wiki/Conway_circle_theorem

% https://en.wikipedia.org/wiki/Carnot%27s_theorem_(inradius,_circumradius)

% https://en.wikipedia.org/wiki/Six_circles_theorem

% https://en.wikipedia.org/wiki/Seven_circles_theorem

% https://en.wikipedia.org/wiki/Schinzel%27s_theorem

% https://en.wikipedia.org/wiki/Monge%27s_theorem

% https://en.wikipedia.org/wiki/Bundle_theorem

% https://en.wikipedia.org/wiki/Five_circles_theorem

% https://en.wikipedia.org/wiki/Descartes%27_theorem

% https://en.wikipedia.org/wiki/Lester%27s_theorem

% https://en.wikipedia.org/wiki/Miquel%27s_theorem

% https://en.wikipedia.org/wiki/Van_Schooten%27s_theorem

% https://en.wikipedia.org/wiki/Trillium_theorem

% https://en.wikipedia.org/wiki/Th%C3%A9bault%27s_theorem

% https://en.wikipedia.org/wiki/Reuschle%27s_theorem

% https://en.wikipedia.org/wiki/Pompeiu%27s_theorem

% % https://en.wikipedia.org/wiki/Garfield%27s_proof_of_the_Pythagorean_theorem

\chapter{Konstrukcje z cyrklem i linijką}
Do napisania...

\bibliography{geo-textbook}{}
\bibliographystyle{plain}

\raggedright
\indexprologue{\small Tekst prologu...}
\printindex

\indexprologue{\small Tekst prologu...}
\printindex[persons]

\end{document}

Neugebauer zrobione:
101-105