\documentclass{greaseproof}
\begin{document}

\thispagestyle{empty}

\newpage~\newpage~\newpage~

\thispagestyle{empty}

\begin{center}
    \Huge{\textbf{Zarys teorii geometrycznych}\\}
    \huge{Epafrodyt z Ptolemais,\\
    Lachmid z Al-Hiry,\\
    Achradyna Ecetryjska}
\end{center}

\tableofcontents
\cleardoublepage

\subsection{Aksjomaty Euklidesa}

\begin{euclidelement}[opisanie 1.1]
	Punktem lub znakiem jest, co nie ma żadnych części, lub co nie ma żadnej wielkości.
\end{euclidelement}

\begin{euclidelement}[opisanie 1.11]
	Kąt rozwarty jest ten, który jest większy od kąta prostego.
\end{euclidelement}

\begin{euclidelement}[opisanie 1.21]
	Figury trójkątne prostokreślne są te, które są ograniczone trzema liniami prostymi.
\end{euclidelement}

% TODO: https://kpbc.umk.pl/dlibra/publication/37/edition/66/content

{\color{red}
Euklides wyróżnił kilka pojęć pierwotnych (takich jak punkt, który był dla Euklidesa \emph{tym, co nie ma żadnych części}) i pięć aksjomatów, przytoczonych za książką \emph{O Elementach Euklidesa}:

\begin{enumerate}
	\item Zakłada się, że od każdego punktu do każdego punktu można poprowadzić linię prostą.
	\item I że ograniczoną prostą można ciągle przedłużać po prostej.
	\item I że z każdego środka każdym rozwarciem można zakreślić kolo.
	\item I że wszystkie kąty proste są równe między sobą.
	\item I jeżeli prosta padająca na dwie proste tworzy po jednej stronie kąty wewnętrzne, które w sumie są mniejsze od dwóch prostych, to te proste przedłużone nieograniczenie schodzą się po tej stronie, po której kąty te w sumie są mniejsze od dwóch prostych.
\end{enumerate}

Pojęcia pierwotne i aksjomaty Euklidesa nie są jednak idealne.
Dlatego zamiast nich będziemy używać aksjomatów Hilberta podanych około 1899 roku.
}
\section{Aksjomaty Hilberta}

\section{Twierdzenie Talesa i podobieństwo}

% intercept theorem, also known as Thales's theorem, basic proportionality theorem or side splitter theorem
\begin{theorem}[Talesa]
    Jeśli ramiona kąta płaskiego przetnie się 2 równoległymi prostymi:
    \begin{center}
        \begin{tikzpicture}
            \tkzDefPoint(0, 0.5){O}
            \tkzDefPoint(1.5, 0){A}
            \tkzDefPoint(2, 1){Ap}
            \tkzDefPointBy[homothety=center O ratio 1.618](A) \tkzGetPoint{B}
            \tkzDefLine[parallel=through B](A,Ap) \tkzGetPoint{Bp}
            \tkzInterLL(O,Ap)(B,Bp) \tkzGetPoint{Bpp}
            \tkzDrawPoints[fill=gray,opacity=.9](O,A,B,Ap,Bpp)
            \tkzLabelPoint[above](O){$O$}
            \tkzLabelPoint[below](A){$A$}
            \tkzLabelPoint[below](B){$B$}
            \tkzLabelPoint[above left](Bpp){$B'$}
            \tkzLabelPoint[above left](Ap){$A'$}
            \tkzDrawLine[thick](O,B)
            \tkzDrawLine[thick](O,Bpp)
            \tkzDrawLine[color=blue, thick](A,Ap)
            \tkzDrawLine[color=blue, thick](B,Bpp)
        \end{tikzpicture}
        \end{center}
    to długości odcinków wyznaczonych przez te proste na jednym z ramion kąta są proporcjonalne do długości odpowiednich odcinków na drugim ramieniu kąta, a zatem
    \begin{equation}
        \frac{|OA|}{|OB|} = \frac{|OA'|}{|OB'|} = \frac{|AA'|}{|BB'|}.
    \end{equation}
\end{theorem}

Tradycja przypisuje jego sformułowanie Talesowi z Miletu, chociaż znane było starożytnym Babilończykom i Egipcjanom.
\index[persons]{Tales z Miletu}%
Pierwszy znany dowód pojawia się w Elementach Euklidesa.

\begin{proof}
    Do napisania.
\end{proof}

\subsection{Podobieństwo.} Lorem ipsum dolor sit amet, consectetur adipiscing elit, sed do eiusmod tempor incididunt ut labore et dolore magna aliqua. Ut enim ad minim veniam, quis nostrud exercitation ullamco laboris nisi ut aliquip ex ea commodo consequat. Duis aute irure dolor in reprehenderit in voluptate velit esse cillum dolore eu fugiat nulla pariatur. Excepteur sint occaecat cupidatat non proident, sunt in culpa qui officia deserunt mollit anim id est laborum.
Cechy przystawania/podobieństwa.
Skala podobieństwa.
\subsection{Twierdzenie Ptolemeusza.}
Klaudiusz Ptolemeusz był astronomem, matematykiem i~geografem pochodzenia greckiego.
Urodzon w Tebaidzie (około roku 100), kształcił się, działał w~Aleksandrii; tam też zmarł około roku 170.
Napisał po grecku Μαθηματικὴ Σύνταξις, traktat w trzynastu księgach znany lepiej jako Almagest zawierający kompendium wiedzy astronomicznej oraz matematyczny wykład teorii geocentrycznej.
Tam też znajduje się następujące twierdzenie:

\begin{theorem}[Ptolemeusza, 140]
    W czworokącie wypukłym $ABCD$ zachodzi
    \begin{equation}
        |AC| \cdot |BD| \le |AB| \cdot |CD| + |BC| \cdot |AD|,
    \end{equation}
    z równością wtedy i tylko wtedy, gdy na czworokącie $ABCD$ można opisać okrąg.
\end{theorem}

O twierdzeniu Ptolemeusza piszą Bogdańska, Neugebauer \cite[s. 62, 63]{neugebauer_2018}.
Wynika z niego:

\begin{theorem}[Carnot, 1???]
    Niech $ABC$ będzie trójkątem wpisanym w okrąg o środku $O$ i promieniu $R$ oraz opisanym na okręgu o promieniu $r$.
    Oznaczmy przez $OO_A$ (i analogicznie $OO_B$, $OO_C$) znakowaną odległość punktu $O$ od boku $BC$.
    Wtedy 
    \begin{equation}
        OO_A + OO_B + OO_C = R + r.
    \end{equation}
    (Odległość jest ujemna wtedy i tylko wtedy, gdy cały odcinek leży poza trójkątem).
\end{theorem}
\subsection{Potęga punktu względem okręgu.} Lorem ipsum dolor sit amet, consectetur adipiscing elit, sed do eiusmod tempor incididunt ut labore et dolore magna aliqua. Ut enim ad minim veniam, quis nostrud exercitation ullamco laboris nisi ut aliquip ex ea commodo consequat. Duis aute irure dolor in reprehenderit in voluptate velit esse cillum dolore eu fugiat nulla pariatur. Excepteur sint occaecat cupidatat non proident, sunt in culpa qui officia deserunt mollit anim id est laborum.
\begin{proposition}[twierdzenie o siecznych i stycznych]
	Jeżeli...
\end{proposition}
\begin{definition}[potęga punktu względem okręgu]
	Jeżeli...
\end{definition}
\begin{proposition}[potęgowe kryterium współokręgowości]
	Jeżeli...
\end{proposition}
\begin{definition}[oś potęgowa]
	Jeżeli...
\end{definition}
\begin{theorem}[Monge'a]
	Jeżeli...
\end{theorem}
\begin{theorem}[Auberta]
	Jeżeli...
\end{theorem}

\subsection{Dwusieczna w trójkącie.} Lorem ipsum dolor sit amet, consectetur adipiscing elit, sed do eiusmod tempor incididunt ut labore et dolore magna aliqua. Ut enim ad minim veniam, quis nostrud exercitation ullamco laboris nisi ut aliquip ex ea commodo consequat. Duis aute irure dolor in reprehenderit in voluptate velit esse cillum dolore eu fugiat nulla pariatur. Excepteur sint occaecat cupidatat non proident, sunt in culpa qui officia deserunt mollit anim id est laborum.
\begin{proposition}[twierdzenie o dwusiecznej]
	Jeżeli...
\end{proposition}
\begin{theorem}[Lehmusa-Steinera]
	Jeżeli...
\end{theorem}
\begin{definition}[okrąg Apoloniusza]
	Jeżeli...
\end{definition}

\subsection{Okręgi ortogonalne i pęki okręgów.} Lorem ipsum dolor sit amet, consectetur adipiscing elit, sed do eiusmod tempor incididunt ut labore et dolore magna aliqua. Ut enim ad minim veniam, quis nostrud exercitation ullamco laboris nisi ut aliquip ex ea commodo consequat. Duis aute irure dolor in reprehenderit in voluptate velit esse cillum dolore eu fugiat nulla pariatur. Excepteur sint occaecat cupidatat non proident, sunt in culpa qui officia deserunt mollit anim id est laborum.
\begin{theorem}[Ponceleta]
	Jeżeli...
\end{theorem}

\subsection{Twierdzenia Eulera i Morleya.} Lorem ipsum dolor sit amet, consectetur adipiscing elit, sed do eiusmod tempor incididunt ut labore et dolore magna aliqua. Ut enim ad minim veniam, quis nostrud exercitation ullamco laboris nisi ut aliquip ex ea commodo consequat. Duis aute irure dolor in reprehenderit in voluptate velit esse cillum dolore eu fugiat nulla pariatur. Excepteur sint occaecat cupidatat non proident, sunt in culpa qui officia deserunt mollit anim id est laborum.
\textbf{Prosta Eulera}.
\textbf{Okrąg dziewięciu punktów}.

\section{Trygonometria}
\subsection{Trygonometria}
\subsubsection{Prawo sinusów}
$$\frac{a}{\sin \alpha} = \frac{b}{\sin \beta} = \frac{c}{\sin \gamma} = 2R$$
% https://en.wikipedia.org/wiki/Law_of_sines

\subsubsection{Prawo cosinusów}
$$c^2 = a^2 + b^2 - 2ab \cos \gamma$$
% https://en.wikipedia.org/wiki/Law_of_cosines

% https://en.wikipedia.org/wiki/Law_of_tangents

\subsubsection{Rozwiązywanie trójkątów}
Wzór Mollweide'a.
\index{wzór!Mollweide'a}%
Problem Hansena
\index{problem!Hansena}%
Problem Snelliusa-Pothenota.
\index{problem!Snelliusa-Pothenota}%
% https://en.wikipedia.org/wiki/Mollweide%27s_formula
% https://en.wikipedia.org/wiki/Snellius%E2%80%93Pothenot_problem
% https://en.wikipedia.org/wiki/Hansen%27s_problem

\section{Geometrie nieeuklidesowe}
Aksjomat Arystotelesa.
Lemat Proklusa.
Aksjomat Claviusa.
Aksjomat Clairauta.
Aksjomat Simsona.
Aksjomat Playfaire'a.
Aksjomat Wallisa.
Aksjomat Bolyi.
Czworokąt Saccheriego.
Aksjomat Legendre'a.
Model Poincarego.
Geometria hiperboliczna.

\section{Konstrukcje geometryczne}
Konstruowalna => stopień Q(x) nad Q to potęga 2, ale nie w drugą stronę.
Podwojenie sześcianu.
Trysekcja kąta.
<=: Hartshorne, papierowa strona 245.

17-kąt

\section{Stereometria}
% Hartshorne: rozdział 8
Pięć wielościanów
Cauchy's rigidity theorem % Hartshorne: section 45
Siamese dodecahedron
% https://en.wikipedia.org/wiki/Johnson_solid

\section{Pomieszane}
% https://en.wikipedia.org/wiki/Casey%27s_theorem

\color{red}

\subsection{Twierdzenie Caseya}
WIP: Casey w 1866 roku uogólnił twierdzenie Ptolemeusza.

\color{black}

%
\subsection{Współpękowość, współliniowość}

Znamy trzy twierdzenia o współliniowości: ..., ... i twierdzenie o prostej Auberta ...

\begin{proposition}[twierdzenie Salmona]
	Dany jest okrąg oraz trzy jego różne cięciwy $PA$, $PB$, $PC$ takie, że przekrojem okręgów na średnicach $PA$, $PB$ (odpowiednio: $PB$, $PC$ i $PA$, $PC$) są punkty $P$, $M$ (odpowiednio: $P$, $K$ oraz $P$, $L$).
	Wtedy punkty $K$, $L$, $M$ są współliniowe.
\end{proposition}

\begin{proposition}[twierdzenie Menelaosa]
	...
	Wówczas punkty $K, L, M$ są współliniowe wtedy i tylko wtedy, gdy zachodzi
	\begin{equation}
		[AMB] [BKC] [CLA] = -1.
	\end{equation}
\end{proposition}
% https://en.wikipedia.org/wiki/Menelaus%27s_theorem
It is uncertain who actually discovered the theorem; however, the oldest extant exposition appears in Spherics by Menelaus. In this book, the plane version of the theorem is used as a lemma to prove a spherical version of the theorem.

% \begin{proposition}[twierdzenie Carnota???]
	% Neugebauer, strona 108.
% \end{proposition}

% https://en.wikipedia.org/wiki/Newton%E2%80%93Gauss_line#Existence_of_the_Newton%E2%88%92Gauss_line

\begin{proposition}
	Środki trzech przekątnych czworoboku zupełnego leżą na jednej prostej, zwaną prostą Newtona-Gaussa.
\end{proposition}

\begin{proposition}[twierdzenie Desargues'a]
	Neugebauer, strona 109.
	% https://en.wikipedia.org/wiki/Desargues%27s_theorem
\end{proposition}

\begin{proposition}[twierdzenie Pascala]
	Neugebauer, strona 113.
	% https://en.wikipedia.org/wiki/Pascal%27s_theorem
\end{proposition}

\begin{proposition}[twierdzenie Pappusa]
	Neugebauer, strona 114.
	% https://en.wikipedia.org/wiki/Pappus%27s_hexagon_theorem
\end{proposition}


\begin{problem}[zadanie Napoleona]
	Podzielić dany okrąg (bez znanego środka) na cztery łuki równej miary korzystając z cyrkla, ale nie linijki.
\end{problem}

Nie wiadomo, czy Napoleon wymyślił albo rozwiązał przedstawione wyżej zadanie konstrukcyjne.
Rozwiązanie: \cite[s. 116]{neugebauer} z wykorzystaniem okręgów Torricelliego.
\index{okrąg Torricelliego}%

\begin{problem}[zadanie Fermata]
	Dany jest trójkąt $ABC$.
	Znaleźć punkt $F$ taki, by suma $|FA| + |FB| + |FC|$ była możliwie najmniejsza.
\end{problem}

Powyższe zadanie rozwiązał Evangelista Torricelli, który dostał je w formie wyzwania od Fermata.
Rozwiązanie opublikował student Torricelliego, Viviani, w 1659 roku.
% TODO: Johnson, R. A. Modern Geometry: An Elementary Treatise on the Geometry of the Triangle and the Circle. Boston, MA: Houghton Mifflin, pp. 221-222, 1929.

Uogólnieniem twierdzenia o współpękowości symetralnych boków trójkąta jest:

\begin{proposition}[twierdzenie Carnota]
	... wtedy i tylko wtedy, gdy
	\begin{equation}
		|AM|^2 + |BK|^2 + |CL|^2 = |AL|^2 + |CK|^2 + |BM|^2.
	\end{equation}
\end{proposition}

Ważnym kryterium współpękowości trzech czewian jest:

\begin{proposition}[twierdzenie Cevy (1678)]
	Dany jest trójkąt $ABC$ i trzy różne od wierzchołków punkty $K \in BC$, $L \in CA$, $M \in AB$.
	Wówczas czewiany $AK$, $BL$, $CM$ są współpękowe wtedy i tylko wtedy, gdy
	\begin{equation}
		[AMB] [BKC] [CLA] = 1.
	\end{equation}
\end{proposition}

\textbf{Do zrobienia: czewiany Gergonne'a}.
\index{czewiany Gergonne'a}

\textbf{Do zrobienia: czewiany Nagela}.
\index{czewiany Nagela}



% TODO: rozwiązanie https://en.wikipedia.org/wiki/Napoleon%27s_problem

% CARNOT
Uogólnieniem twierdzenia o współpękowości symetralnych boków trójkąta jest:

\begin{proposition}[twierdzenie Carnota]
	... wtedy i tylko wtedy, gdy
	\begin{equation}
		|AM|^2 + |BK|^2 + |CL|^2 = |AL|^2 + |CK|^2 + |BM|^2.
	\end{equation}
\end{proposition}
% CARNOT

% CEVA
Ważnym kryterium współpękowości trzech czewian jest:

\begin{proposition}[twierdzenie Cevy (1678)]
	Dany jest trójkąt $ABC$ i trzy różne od wierzchołków punkty $K \in BC$, $L \in CA$, $M \in AB$.
	Wówczas czewiany $AK$, $BL$, $CM$ są współpękowe wtedy i tylko wtedy, gdy
	\begin{equation}
		[AMB] [BKC] [CLA] = 1.
	\end{equation}
\end{proposition}

\textbf{Do zrobienia: czewiany Gergonne'a}.
\index{czewiany Gergonne'a}

\textbf{Do zrobienia: czewiany Nagela}.
\index{czewiany Nagela}
% CEVA

% BRIANCHON
Twierdzenie Brianchona, Newtona.
% BRIANCHON

% VAN AUBEL
Twierdzenie van Aubela.
Wzór Routha, równosć Gergonne'a.
% VAN AUBEL

% STEINER
Twierdzenie Steinera.
% STEINER

% LEMOINE
Czewiany, symediany, punkt Lemoine'a.
% LEMOINE
%

\begin{problem}[zadanie Malfattiego]
\index{zadanie Malfattiego}%
	Dany jest trójkąt $\triangle ABC$.
	Skonstruować takie trzy parami styczne okręgi $K_A, K_B, K_C$, że okrąg $K_A$ (odpowiednio: $K_B$, $K_C$) jest wpisany w~kąt $\angle A$ (odpowiednio: $\angle B$, $\angle C$).
\end{problem}

Problem był rozważany na długo przed Malfattim, zajmował się nim matematyk japoński Ajima Naonobu w~XVIII wieku, a~jeszcze wcześniej pojawił się w~rękopisie Gilio di Cecco da Montepulciano z 1384 roku.
\index[persons]{Ajima, Naonobu}%
Malfatti \cite{malfatti_1803} wyprowadził w~1803 roku następujące: niech $r$ będzie promieniem koła wpisanego w~trójkąt, $s$ połową jego obwodu, a~$d_A, d_B, d_C$ odległościami wierzchołków $A, B, C$ od środka koła wpisanego, wtedy promienie kół Malfattiego wyrażają się wzorami
\index[persons]{Malfatti, Gian Francesco}%
\begin{align}
	r_1 & = {\frac {r}{2(s-a)}}(s-r+d_A-d_B-d_C), \\
	r_2 & = {\frac {r}{2(s-b)}}(s-r-d_A+d_B-d_C), \\
	r_3 & = {\frac {r}{2(s-c)}}(s-r-d_A-d_B+d_C).
\end{align}

Prostą konstrukcję okręgów opartą na dwustycznych zawdzięczamy Steinerowi \cite{steiner_1826} w~1826 roku;
\index[persons]{Steiner, Jakob}%
inne rozwiązania podali Lehmus \cite{lehmus_1819}, Catalan \cite{catalan_1846}, Adams \cite{adams_1846}, Derousseau \cite{derousseau_1895}, Pampuch \cite{pampuch_1904}.
% TODO: po poprawie bibliografii, podać tu index persons
(O problemie pisze też Neugebauer \cite[s. 102]{neugebauer_2018}).

Malfatti postawił tak naprawdę inny problem: znalezienia trzech rozłącznych kół zawartych w~trójkącie, których suma pól jest maksymalna i~błędnie założył, że opisane wyżej okręgi stanowią rozwiązanie.
Pomyłkę zauważyli najpierw bez dowodu Lob, Richmond \cite{lob_richmond_1930} w~1930 roku; rygorystyczny dowód, że poprawne rozwiązanie daje algorytm zachłanny podano w~latach sześćdziesiątych tego samego wieku.
% TODO: po poprawie bibliografii, podać tu index persons
% TODO: Lob, H.; Richmond, H. W. (1930), "On the Solutions of Malfatti's Problem for a Triangle", Proceedings of the London Mathematical Society, 2nd ser., 30 (1): 287–304, doi:10.1112/plms/s2-30.1.287.

%
\subsection{Pons asinorum}
In geometry, the theorem that the angles opposite the equal sides of an isosceles triangle are themselves equal is known as the pons asinorum Latin for "bridge of asses", or more descriptively as the isosceles triangle theorem. The theorem appears as Proposition 5 of Book 1 in Euclid's Elements[1]. Its converse is also true: if two angles of a triangle are equal, then the sides opposite them are also equal. 

Dowody:
- Euclid and Proclus
% https://en.wikipedia.org/wiki/Pons_asinorum#Euclid_and_Proclus

- Pappus
% https://en.wikipedia.org/wiki/Pons_asinorum#Pappus

- współcześni, Legendre, Garfield
% https://en.wikipedia.org/wiki/Pons_asinorum#Others
%

\begin{proposition}[twierdzenie o~sześciu okręgach]
\index{twierdzenie!o sześciu okręgach}%
    Dany są trójkąt $\triangle ABC$ oraz okręgi $K_1$, $K_2$, \ldots, $K_7$ zawarte w~tym trójkącie, wpisane kolejno w~kąty $\angle A$, $\angle B$, $\angle C$, $\angle A$, $\angle B$, $\angle C$, $\angle A$ takie, że okręgi $K_i$ oraz $K_{i+1}$ dla $i = 1, 2, \ldots, 6$ są styczne.
    Wtedy $K_1 = K_7$.
\end{proposition}

(Neugebauer \cite[s. 101]{neugebauer_2018} nazywa to twierdzeniem o~siódmym okręgu).
Tabacznikow, Iwanow \cite{ivanov_tabachnikov_2016} pokazali, że jeśli osłabimy założenia: okręgi nie muszą zawierać się w~trójkącie i~wystarczy, że będą styczne do prostych zawierających boki trójkąta, to nadal ciąg okręgów jest od pewnego miejsca okresowy z okresem równym sześć, ale osiągnięcie tego stanu może wymagać dowolnie wielu kroków.
\index[persons]{Tabacznikow, Siergiej (Табачников, Сергей Львович)}%
\index[persons]{Iwanow, Denis (Иванов, Денис)}%

\begin{proof}
    Evelyn, Money-Coutts, Tyrrell \cite[s. 49–58]{evelyn_money_coutts_tyrrell_1974}.
\index[persons]{Evelyn, Cecil John Alvin}%
\index[persons]{Money-Coutts, Godfrey Burdett}%
\index[persons]{Tyrrell, John Alfred}%
\end{proof}

%

% https://en.wikipedia.org/wiki/Taylor_circle
\subsection{Okrąg Taylora}
{
    \emph{WIP: Taylor w 1882 roku zauważył, że rzuty spodków wysokości na pozostałe boki leżą na jednym okręgu.}
}


\section{Geometria -- Uniwersytet Warszawski}
\subsection{Geometria I}
\subsubsection{Geometria 1.1}
Przystawanie figur na płaszczyźnie. % https://en.wikipedia.org/wiki/Congruence_(geometry)
Cechy przystawania trójkątów. % https://en.wikipedia.org/wiki/Congruence_(geometry)#Determining_congruence
Własności równoległoboków.
Problem Fagnano i problem Fermata.
Kąty w okręgu: wpisane, kąty środkowe i kąty dopisane.
Twierdzenia o kątach wpisanych, kątach środkowych i kątach dopisanych do okręgu.
Kątowe warunki na istnienie okręgu przechodzącego przez cztery punkty.
Zastosowanie: okrąg dziewięciu punktów, twierdzenie o prostej Simsona.
Styczna do okręgu, okrąg wpisany w kąt.
Okrąg wpisany w trójkąt, okręgi dopisane do trójkąta.
Warunki istnienia okręgu stycznego do czterech prostych.

\subsubsection{Geometria 1.2}
Stosunek podziału wektora.
% Twierdzenie Talesa, twierdzenie odwrotne i jego zastosowania.
Pole.
Pola wybranych figur, twierdzenie Pitagorasa.
Pole zorientowane.
Twierdzenie Newtona: środek okręgu wpisanego w czworokąt i środki przekątnych tego czworokąta są współliniowe.
Twierdzenie Gaussa: środki przekątnych czworokąta zupełnego są współliniowe.
Definicja jednokładności, podobieństwo figur.
Cechy podobieństwa trójkątów.
Stosunek pól figur podobnych.
Iloczynowe warunki istnienia okręgu przechodzącego przez cztery punkty.
Pojęcie potęgi punktu
względem okręgu.
Twierdzenie Ptolemeusza.

\subsubsection{Geometria 1.3}
Wielkości miarowe w trójkącie: wzór Herona, wzory na promienie okręgów wpisanych, dopisanych.
Twierdzenie o dwusiecznej i okrąg Apoloniusza.
Twierdzenie Cevy (wraz z trygonometryczną wersją), przykłady punktów szczególnych trójkąta: punkt Nagela, punkt Gergonne'a, punkt Lemoine'a.
Punkty izogonalnie sprzężone w trójkącie.
Twierdzenie Menelausa.

\subsubsection{Jednokładność}
Jednokładność.

Konstrukcja obrazu jednokładnego punktu, okręgu, prostej.

Środek jednokładności dwóch trójkątów.

Środki jednokładności dwóch okręgów.

Prosta Eulera w trójkącie (środek okręgu opisanego, środek ciężkości, ortocentrum).

Zastosowanie: Twierdzenie Pascala.

Twierdzenie Kirkmana: jeśli część wspólna dwóch trójkątów wpisanych w okrąg jest sześciokątem wypukłym, to główne przekątne tego sześciokąta przecinają się w jednym punkcie.

Grupa dylatacji na płaszczyźnie.

Twierdzenia o składaniu jednokładności i przesunięć, twierdzenie o środkach jednokładności trzech okręgów.

\subsubsection{Izometrie}
5.
Grupa izometrii na płaszczyźnie.
Konstrukcja obrazu punktu, okręgu, prostej przy translacji, obrocie i symetrii osiowej.
Złożenie dwóch i złożenie trzech symetrii osiowych.
Twierdzenia o składaniu izometrii.
Klasyfikacja izometrii na płaszczyźnie.
Izometrie parzyste i izometrie nieparzyste.
Twierdzenie o redukcji.
Twierdzenie Napoleona: środki ciężkości trójkątów równobocznych zbudowanych na bokach dowolnego trójkąta są wierzchołkami trójkąta równobocznego.
% TODO: https://en.wikipedia.org/wiki/Napoleon%27s_theorem

\subsubsection{Podobieństwa}
6.
Grupa podobieństw płaszczyzny.
Podobieństwa spiralne i odbicia dylatacyjne.
Klasyfikacja podobieństw płaszczyzny.



\subsection{Geometria II UW}
1. Potęga punktu względem okręgu, oś potęgowa dwóch okręgów, środek potęgowy trzech okręgów, twierdzenie Brianchona, konstrukcja stycznej do okręgu samą linijką, okręgi współpękowe, twierdzenie Gaussa-Bodenmillera, twierdzenie o motylku, formuła Eulera na odległość między środkami okręgu opisanego i wpisanego (dla trójkąta), twierdzenie Ponceleta dla trójkąta.

2. Obrazy inwersyjne okręgów i prostych, konforemność inwersji, okręgi stałe inwersji, okręgi prostopadłe, zmiana odległości przy inwersji, twierdzenie Ptolemeusza, zmiana promienia okręgu przy inwersji, łańcuchy Steinera, formuła Kartezjusza, formuła Fussa dla czworokątów, twierdzenie Feuerbacha.

3. Ogniska elipsy i hiperboli, ognisko, kierownica i mimośród stożkowych, asymptoty hiperboli, konstrukcja stycznej do stożkowej, rzuty ustalonego ogniska na styczne, własności izogonalne stożkowych, równania kanoniczne stożkowych, elipsa jako przekrój walca. Ognisko, kierownica i mimośród stożkowej na przekroju stożka. Przekroje stożków ze sferami wpisanymi. Równanie ogólne stożkowej w układzie współrzędnych, duży i mały wyznacznik. Równania stożkowych we współrzędnych biegunowych.

4. Grupa przekształceń afinicznych od strony geometrycznej: powinowactwa osiowe, rozkład przekształcenia afinicznego na podobieństwo i powinowactwo osiowe, kierunki główne przekształcenia afinicznego, niezmienniczość stosunku pól przy przekształceniu afinicznym, obraz okręgu przy przekształceniu afinicznym
Literatura: 	

1. E. H. Askwith, D.D. A Course of Pure Geometry, Cambridge 1917.
2. H. Fukagawa, D. Pedoe, Japanese temple geometry problems. Sangaku, Charles Babbage Research Centre, Winnipeg 1989.
3. R. A. Johnson, Advanced Euclidean geometry: An elementary treatise on the geometry of the triangle and the circle, Dover Publications, Inc., New York 1960.
4. W. Pompe, Geometria kół, Wydawnictwo Szkolne OMEGA, Kraków 2019.
5. V. Prasolov, Zadaczi po planimietrii. Tom I-II (ros.), Nauka, Moskwa 1991


\subsection{Geometria III}
Geometria rzutowa: ujęcie od strony geometrycznej. Płaszczyzna rzutowa (rzeczywista), przekształcenia rzutowe prostych, pęków, stożkowych, pęków stycznych do stożkowych.
Twierdzenia Desarguesa, Pappusa, Pascala, Brianchona.
Dualność: biegun i biegunowa względem okręgu i stożkowych. Sprzężenie biegunowe. Inwolucje rzutowe, twierdzenia inwolucyjne. Pęki okręgów i stożkowych jako generatory inwolucji. Twierdzenie Ponceleta. Stożkowe w ujęciu rzutowym, twierdzenia Steinera i Braikenridge'a-Maclaurina. Rzutowe określenie ogniska i kierownicy stożkowych. Punkty urojone przecięcia prostej ze stożkową w ujęciu czysto geometrycznym.


\section{W przygotowaniu}

\section{Do zrobienia}
Potęga punktu względem okręgu?
Twierdzenie o odcinku środkowym: odcinek łączący środki dwóch boków trójkąta jest równoległy do podstawy i ma połowę jej długości.
Symetria osiowa.
Symetralna: przecinają się w jednym punkcie.
Okrąg.
Styczne, sieczne.
Twierdzenia geometrii koła o miarach kątów.
Twierdzenie Apoloniusza.
Czworokąt cykliczny.
Twierdzenie o prostej Wallace'a-Simsona.
Ortocentrum i trójkąt ortyczny.
Twierdzenie Miquela.
Twierdzenie Pitagorasa.
Twierdzenie Varignona.
Podobieństwo, skala.
Twierdzenie Ptolemeusza.
Twierdzenie Carnota.
Sieczne i styczne.
Potęga punktu względem okręgu.
Twierdzenie o prostej Auberta.
Twierdzenie o dwusiecznej.
Twierdzenie o okręgu Apoloniusza.
Dwustosunek.
Pęki okręgów.
twierdzenia:
- Ptolemeusza
- trójkąty
twierdzenie Pitagorasa, wzór Herona (uogólnienie do czworokątów itd.), twierdzenie Carnota
okrąg opisany, wpisany, ortrocentrym, środek ciężkości
prosta Eulera?
okrąg Feuerbacha?
punkt Torricellego = punkt Fermata
- czworokąty:
opisany/wpisany na okręgu, 
twierdzenia Newtona/Brianchona
nierówności:
- Mikołaja z Kuzy: sinx / x < 2 + cos x / 3 (Guzicki, s. 390)
- Eulera (R >= 2r), izoperymetryczna (S <= pp/3sqrt3), Mitrinovica (r <= ... <= R/2), Leibniza (aa + bb + cc <= 9RR), Weitzenbocka (aa + bb + cc >= 4sqrt 3 S)
Konstrukcje z cyrklem i linijką:
- wielokątów (3, 4, 6, 5, 17, 257, ...)
- okręgi Apoloniusza
Inwersja, Feuerbach.
Okręgi Apoloniusza: \cite[s. 444-461]{guzicki_2021}.
Dwustosunek.
Izometrie, punkty stałe.
Translacje, symetrie osiowe, symetrie środkowe, obroty.
Twierdzenie Chasles'a: każda izometria płaszczyzny jest złożeniem co najwyżej trzech symetrii osiowych.
Symetria osiowa z poślizgiem.
Słowo Banacha.
Klasyfikacja podobieństw.
Okrąg siedmiu punktów. % https://mathworld.wolfram.com/BrocardCircle.html ?
Przekształcenia afiniczne i rzutowe.
% https://www.cut-the-knot.org/Curriculum/Geometry/HeronsProblem.shtml
% This one is a basic optimization problem. It's quite famous, being discussed in Heron's Catoptrica (On Mirrors from the Greek word Katoptron Catoptron = Mirror) that, in all likelihood, saw the light of day some 2000 years ago.
Pitagorasa % https://en.wikipedia.org/wiki/Pythagorean_theorem
% https://en.wikipedia.org/wiki/Spiral_of_Theodorus
Twierdzenie Ponceleta.
Prosta/twierdzenie Eulera.
Twierdzenie Morleya
Okrąg dziewięciu punktóws
Trygonometria - sinusów, cosinusów, Stewarta.
Wzór Herona.
Wzór Brahmagupty
Twierdzenie Urquharta
Punkt i kąt Crelle'a-Brocarda
Aksjomaty. Kąty naprzemianległe i odpowiadające.
Przystawanie trójkątów.
Łamane i wielokąty.
Równoległobok.
Równoważność wektorów.
Symetria osiowa.
Symetralna.
Styczna do okręgu.
gnomon % https://en.wikipedia.org/wiki/Theorem_of_the_gnomon
Kąty środkowe i wpisane.
Cykliczność. Prosta Wallace'a.
Ortocentrum i trójkąt ortyczny.
Twierdzenie Miquela.
Dwusieczna. Okrąg wpisany i dopisane.
Twierdzenie Pitagorasa.

Podobieństwo.
Twierdzenie Ptolemeusza.
Twierdzenie Carnota.
Potęga punktu względem okręgu.
Pęki okręgów.
Twierdzenie Eulera.
Twierdzenie Morleya.
Trygonometria. Wzór Herona.
Twierdzenie Urquharta.
Kąt Crelle'a-Brocarda.
Twierdzenie o siódmym okręgu.
Współliniowość.
Współpękowość.
Ceva i Menelaos.
Twierdzenie Ponceleta.
Jednokładność.
Inwersja.
Dwustosunek.

W 1822 roku Karl Wilhelm Feuerbach, którego nazwiskiem nazywa się czasem okrąg dziewięciu punktów, zauważył, że sześć charakterystycznych punktów trójkąta – środki boków oraz spodki wysokości – leżą na wspólnym okręgu. Odkrycia tego dokonali wcześniej, w 1821 roku, Charles Brianchon i Jean-Victor Poncelet[3]. Jeszcze wcześniej, nad współokręgowością wspomnianych punktów zastanawiali się Benjamin Bevan (1804) i John Butterworth (1807)[3].
Krótko po Feuerbachu, matematyk Olry Terquem niezależnie udowodnił istnienie okręgu i jako pierwszy zauważył, że leżą na nim również środki odcinków łączących wierzchołki z ortocentrum. Terquem jako pierwszy użył również nazwy „okrąg dziewięciu punktów”[4].
Karl Wilhelm Feuerbach udowodnił, że w dowolnym trójkącie okrąg dziewięciu punktów jest styczny wewnętrznie do okręgu wpisanego i zewnętrznie do trzech okręgów dopisanych[5]. Punkt styczności okręgu wpisanego i okręgu dziewięciu punktów nazywa się często punktem Feuerbacha[6].
% Środek okręgu dziewięciu punktów leży na tzw. prostej Eulera, dokładnie w połowie odcinka pomiędzy ortocentrum tego trójkąta a środkiem okręgu na nim opisanego[7].
In geometry, the nine-point circle is a circle that can be constructed for any given triangle. It is so named because it passes through nine significant concyclic points defined from the triangle. These nine points are:
The midpoint of each side of the triangle
The foot of each altitude
The midpoint of the line segment from each vertex of the triangle to the orthocenter (where the three altitudes meet; these line segments lie on their respective altitudes).[1][2]
The nine-point circle is also known as Feuerbach's circle (after Karl Wilhelm Feuerbach), Euler's circle (after Leonhard Euler), Terquem's circle (after Olry Terquem), the six-points circle, the twelve-points circle, the n-point circle, the medioscribed circle, the mid circle or the circum-midcircle. Its center is the nine-point center of the triangle.[3][4]
Although he is credited for its discovery, Karl Wilhelm Feuerbach did not entirely discover the nine-point circle, but rather the six-point circle, recognizing the significance of the midpoints of the three sides of the triangle and the feet of the altitudes of that triangle. (See Fig. 1, points D, E, F, G, H, I.) (At a slightly earlier date, Charles Brianchon and Jean-Victor Poncelet had stated and proven the same theorem.) But soon after Feuerbach, mathematician Olry Terquem himself proved the existence of the circle. He was the first to recognize the added significance of the three midpoints between the triangle's vertices and the orthocenter. (See Fig. 1, points J, K, L.) Thus, Terquem was the first to use the name nine-point circle.
% The first major discovery that led to the discovery of the nine-point circle was by Benjamin Bevan in 1804 as he made a mathematical proposal that inevitably established the conclusions that “the nine point center bisects the distance between the circumcentre and the orthocenter, and that the radius of the nine-point circle is half the radius of the circumcircle”(Mackay, "History of the Nine Point Circle"). A mathematician by the name of John Butterworth later in 1804 proved this proposal and subsequent conclusions in mathematical journals, and in 1807 formed a key question for the further exploration of Benjamin Bevan’s proposed phenomenon.  He asks, “When the base and vertical angle are given, what is the locus of the centre of the circle passing through the three centres of the circles touching one side and the prolongation of the other two sides of a plane triangle?” in 1806.  In response a man by the name of John Whitley made the important discovery that the circumcircle of a triangle intersects two of the midpoints of the sides, two of the feet of the altitudes of the triangle, as well as two of the mid points of the segments intercepted between the orthocenter and the vertices.  At this point in time only seven of the nine points had been discovered. The discovery of the full nine-points and the full nine points were fully mentioned for the first time in 1821 by Jean-Victor Poncelet and his partner Bianchon in a mathematical journal.  Soon after in 1822 Karl Feurbach proved the existence of the same circle independently and received much of the credit for its discovery.  Up until this point in time there was no official name for this circle that had been discovered but in 1842 a man by the name of Olry Terquem coined the term the nine-point circle in an analytical proof investigating some of the subsequent properties of the circle.  Today we know of at least 25 important points that actually lie on the so called "Nine point circle" (Mackay, "History of the Nine Point Circle"). 
% The nine-point circle also passes through Kimberling centers X_i for i=11 (the Feuerbach point), 113, 114, 115 (center of the Kiepert hyperbola), 116, 117, 118, 119, 120, 121, 122, 123, 124, 125 (center of the Jerabek hyperbola), 126, 127, 128, 129, 130, 131, 132, 133, 134, 135, 136, 137, 138, 139, 1312, 1313, 1560, 1566, 2039, 2040, and 2679.

1821

The nine points are explicitly mentioned in Gergonne's Annales de Mathematiques , volume xi., in an article by Brianchon and Poncelet. This article contains the theorem establishing the characteristic property of the nine point circle.

1822

First enunciation of Feuerbach's Theorem, including the first published proof, appears in Karl Wilhelm Feuerbach's Eigenschaften einiger merkwiirdigen Punkte des geradlinigen Dreiecks, along with many other interesting proofs relating to the nine point circle.

1842

The circle is officially designated the "nine point circle" (le cercle des neuf points) by Terquem, one of the editors of the Nouvelles Annales. (see Volume I page 198). Terquem published the second analytical proof of the theorem that the nine point circle touches the incircle and the excircles.

\bibliography{geo-textbook}{}
\bibliographystyle{plain}

\raggedright
\indexprologue{\small Tekst prologu...}
\printindex

\indexprologue{\small Tekst prologu...}
\printindex[persons]

\end{document}

% https://en.wikipedia.org/wiki/Problem_of_Apollonius
% https://en.wikipedia.org/wiki/Poncelet%E2%80%93Steiner_theorem
% https://en.wikipedia.org/wiki/Compass_equivalence_theorem
% https://en.wikipedia.org/wiki/Angle_trisection
% https://en.wikipedia.org/wiki/Mohr%E2%80%93Mascheroni_theorem
% burdel
% https://en.wikipedia.org/wiki/Pasch%27s_theorem
% https://en.wikipedia.org/wiki/Apollonius%27s_theorem jest specjalnym przypadkiem Stewarta
% https://en.wikipedia.org/wiki/Aristarchus%27s_inequality
% https://en.wikipedia.org/wiki/Ptolemy%27s_inequality
% https://en.wikipedia.org/wiki/Diophantus_II.VIII
% https://en.wikipedia.org/wiki/Pappus%27s_area_theorem

% https://en.wikipedia.org/wiki/Heron%27s_formula
% https://en.wikipedia.org/wiki/Brahmagupta%27s_formula
% https://en.wikipedia.org/wiki/Problem_of_Apollonius
% https://en.wikipedia.org/wiki/Pitot_theorem
% https://en.wikipedia.org/wiki/Brahmagupta_theorem
% https://en.wikipedia.org/wiki/Japanese_theorem_for_cyclic_quadrilaterals
% https://en.wikipedia.org/wiki/Japanese_theorem_for_cyclic_polygons
% https://en.wikipedia.org/wiki/Kosnita%27s_theorem
% https://en.wikipedia.org/wiki/Musselman%27s_theorem
% https://en.wikipedia.org/wiki/Harcourt%27s_theorem
% https://en.wikipedia.org/wiki/Feuerbach_point
% https://en.wikipedia.org/wiki/Euler%27s_theorem_in_geometry
% https://en.wikipedia.org/wiki/Equal_incircles_theorem
% https://en.wikipedia.org/wiki/Conway_circle_theorem
% https://en.wikipedia.org/wiki/Carnot%27s_theorem_(inradius,_circumradius)
% https://en.wikipedia.org/wiki/Six_circles_theorem
% https://en.wikipedia.org/wiki/Seven_circles_theorem
% https://en.wikipedia.org/wiki/Schinzel%27s_theorem
% https://en.wikipedia.org/wiki/Monge%27s_theorem
% https://en.wikipedia.org/wiki/Bundle_theorem
% https://en.wikipedia.org/wiki/Five_circles_theorem
% https://en.wikipedia.org/wiki/Descartes%27_theorem
% https://en.wikipedia.org/wiki/Lester%27s_theorem
% https://en.wikipedia.org/wiki/Miquel%27s_theorem
% https://en.wikipedia.org/wiki/Van_Schooten%27s_theorem
% https://en.wikipedia.org/wiki/Trillium_theorem
% https://en.wikipedia.org/wiki/Th%C3%A9bault%27s_theorem
% https://en.wikipedia.org/wiki/Reuschle%27s_theorem
% https://en.wikipedia.org/wiki/Pompeiu%27s_theorem
% % https://en.wikipedia.org/wiki/Garfield%27s_proof_of_the_Pythagorean_theorem
% https://en.wikipedia.org/wiki/Ptolemy%27s_theorem
% https://en.wikipedia.org/wiki/Casey%27s_theorem
