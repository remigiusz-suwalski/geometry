
\documentclass{parchment}
\begin{document}

% strona pierwsza
\thispagestyle{empty}
{\noindent\fontsize{18pt}{18pt}\selectfont Biblioteka Aleksandryjska, tom I}

\noindent\makebox[\linewidth]{\rule{\textwidth}{1pt}}

\newpage

% strona druga
\thispagestyle{empty}
\phantom{nothing}
\newpage

% strona trzecia
\thispagestyle{empty}
{\noindent\fontsize{18pt}{18pt}\selectfont Epafrodyt z Ptolemais}

\noindent\makebox[\linewidth]{\rule{\textwidth}{1pt}}

\vspace{10mm}

{\noindent\fontsize{24pt}{24pt}\selectfont \textbf{Geometria}}
\vspace{10mm}

{\noindent\fontsize{14pt}{14pt}\selectfont Wydanie zerowe (eksperymentalne)}

\newpage

% strona czwarta
\thispagestyle{empty}
\begin{figure}[H]
\begin{minipage}[b]{.48\linewidth}
{\noindent Epafrodyt z Eudoksos\\
do napisania\\
do napisania\\
do napisania}
\end{minipage}
\begin{minipage}[b]{.48\linewidth}
{\noindent do napisania\\
do napisania\\
do napisania\\
do napisania}
\end{minipage}
\end{figure}

{\noindent \textbf{Kategorie MSC 2020}\\do napisania} \vspace{5mm}

{\noindent \textbf{Tytuł oryginału}\\do napisania} \vspace{5mm}

{\noindent \textbf{Z greki tłumaczyła}\\do napisania} \vspace{5mm}

{\noindent \textbf{Okładkę zaprojektował}\\do napisania} \vspace{5mm}

{\noindent \textbf{Zredagował}\\do napisania} \vspace{5mm}

{\noindent \textbf{Zredagowała technicznie}\\do napisania} \vspace{5mm}

{\noindent \textbf{Złożyli i połamali}\\do napisania} \vspace{5mm}

{\noindent \textbf{Korekty dokonali}\\do napisania} \vfill

{\noindent Copyleft © 2024 by Antykwariat Czarnoksięski.
Książka, a żeby było śmieszniej także każda jej część, mogą być przedrukowywane oraz w jakikolwiek inny sposób reprodukowane czy powielane mechanicznie, fotooptycznie, zapisywane elektronicznie lub magnetycznie, oraz odczytywane w środkach publicznego przekazu bez pisemnej zgody wydawcy.
}

\vspace{5mm}
{
    \noindent
    Tekst udostępniany na licencji Creative Commons: uznanie autorstwa, użycie niekomercyjne. Przeczytaj więcej na \texttt{https://creativecommons.org/licenses/by-nc/4.0/deed.pl}.
}

\vspace{5mm}

{\noindent Przygotowano w systemie \TeX, wydrukowano na siarczystym papierze.}

% strona piąta
\newpage
\section*{Przedmowa}
Do napisania.

\begin{flushright}
Epafrodyt,\\gdzie, kiedy
\end{flushright}

\tableofcontents
% \pagestyle{fancy} % Enable default headers and footers again
\cleardoublepage % Start the following content on a new page

\chapter{Prolog}

\section{Cechy przystawania}
\section{Pons asinorum}

Twierdzenie o odcinku środkowym: odcinek łączący środki dwóch boków trójkąta jest równoległy do podstawy i ma połowę jej długości.

Symetria osiowa.
Symetralna: przecinają się w jednym punkcie.

Okrąg.
Styczne, sieczne.
Twierdzenia geometrii koła o miarach kątów.
Twierdzenie Apoloniusza.

Czworokąt cykliczny.
Twierdzenie o prostej Wallace'a-Simsona.

Ortocentrum i trójkąt ortyczny.

Twierdzenie Miquela.

Twierdzenie Pitagorasa.

\subsection{Twierdzenie Talesa i twierdzenie do niego odwrotne.}

% intercept theorem, also known as Thales's theorem, basic proportionality theorem or side splitter theorem
\begin{theorem}[Talesa]
    Jeśli ramiona kąta płaskiego przetnie się 2 równoległymi prostymi:
    \begin{center}
        \begin{tikzpicture}
            \tkzDefPoint(0, 0.5){O}
            \tkzDefPoint(1.5, 0){A}
            \tkzDefPoint(2, 1){Ap}
            \tkzDefPointBy[homothety=center O ratio 1.618](A) \tkzGetPoint{B}
            \tkzDefLine[parallel=through B](A,Ap) \tkzGetPoint{Bp}
            \tkzInterLL(O,Ap)(B,Bp) \tkzGetPoint{Bpp}
            \tkzDrawPoints[fill=gray,opacity=.9](O,A,B,Ap,Bpp)
            \tkzLabelPoint[above](O){$O$}
            \tkzLabelPoint[below](A){$A$}
            \tkzLabelPoint[below](B){$B$}
            \tkzLabelPoint[above left](Bpp){$B'$}
            \tkzLabelPoint[above left](Ap){$A'$}
            \tkzDrawLine[thick](O,B)
            \tkzDrawLine[thick](O,Bpp)
            \tkzDrawLine[color=blue, thick](A,Ap)
            \tkzDrawLine[color=blue, thick](B,Bpp)
        \end{tikzpicture}
        \end{center}
    to długości odcinków wyznaczonych przez te proste na jednym z ramion kąta są proporcjonalne do długości odpowiednich odcinków na drugim ramieniu kąta, a zatem
    \begin{equation}
        \frac{|OA|}{|OB|} = \frac{|OA'|}{|OB'|} = \frac{|AA'|}{|BB'|}.
    \end{equation}
\end{theorem}

Tradycja przypisuje jego sformułowanie Talesowi z Miletu, chociaż znane było starożytnym Babilończykom i Egipcjanom.
\index[persons]{Tales z Miletu}%
% Pierwszy znany dowód pojawia się w Elementach Euklidesa.
Najstarszy zachowany dowód twierdzenia Talesa zamieszczony jest w VI. księdze Elementów Euklidesa. 
% https://en.wikipedia.org/wiki/Intercept_theorem#Claim_3

\begin{proof}
    Do napisania.
\end{proof}

\begin{proposition}[twierdzenie o odcinku środkowym]
	Jeżeli...
\end{proposition}

\begin{theorem}[Varignona]
	Równoległobok środkowy...
\end{theorem}

Twierdzenie Varignona.
Podobieństwo, skala.

Twierdzenie Ptolemeusza.
Twierdzenie Carnota.

Sieczne i styczne.
Potęga punktu względem okręgu.

Twierdzenie o prostej Auberta.

Twierdzenie o dwusiecznej.
Twierdzenie o okręgu Apoloniusza.

Dwustosunek.
Pęki okręgów.

Twierdzenie Ponceleta.

Prosta/twierdzenie Eulera.

Twierdzenie Morleya
Okrąg dziewięciu punktów

Trygonometria - sinusów, cosinusów, Stewarta.
Wzór Herona.
Wzór Brahmagupty

Twierdzenie Urquharta

Punkt i kąt Crelle'a-Brocarda

Twierdzenie o siódmym okręgu

Twierdzenie Caseya, Taylora.

\chapter{Współliniowość, współpękowość}

Znamy trzy twierdzenia o współliniowości: ..., ... i twierdzenie o prostej Auberta ...

\begin{proposition}[twierdzenie Salmona]
	Dany jest okrąg oraz trzy jego różne cięciwy $PA$, $PB$, $PC$ takie, że przekrojem okręgów na średnicach $PA$, $PB$ (odpowiednio: $PB$, $PC$ i $PA$, $PC$) są punkty $P$, $M$ (odpowiednio: $P$, $K$ oraz $P$, $L$).
	Wtedy punkty $K$, $L$, $M$ są współliniowe.
\end{proposition}

\begin{proposition}[twierdzenie Menelaosa]
	...
	Wówczas punkty $K, L, M$ są współliniowe wtedy i tylko wtedy, gdy zachodzi
	\begin{equation}
		[AMB] [BKC] [CLA] = -1.
	\end{equation}
\end{proposition}

It is uncertain who actually discovered the theorem; however, the oldest extant exposition appears in Spherics by Menelaus. In this book, the plane version of the theorem is used as a lemma to prove a spherical version of the theorem.
% https://en.wikipedia.org/wiki/Menelaus%27s_theorem

\begin{proposition}[twierdzenie Carnota???]
	Neugebauer, strona 108.
\end{proposition}

Twierdzenie GAUSSA???
Środki trzech przekątnych czworoboku zupełnego leżą na jednej prostej.

\begin{proposition}[twierdzenie Desarguesa???]
	Neugebauer, strona 109.
\end{proposition}

\begin{proposition}[twierdzenie Pascala]
	Neugebauer, strona 113.
\end{proposition}

\begin{proposition}[twierdzenie Pappusa]
	Neugebauer, strona 114.
\end{proposition}

\color{red}

\begin{problem}[zadanie Napoleona]
	Podzielić dany okrąg (bez znanego środka) na cztery łuki równej miary korzystając z cyrkla, ale nie linijki.
\end{problem}

Nie wiadomo, czy Napoleon wymyślił albo rozwiązał przedstawione wyżej zadanie konstrukcyjne.
Rozwiązanie: \cite[s. 116]{neugebauer} z wykorzystaniem okręgów Torricelliego.
\index{okrąg Torricelliego}%

\begin{problem}[zadanie Fermata]
	Dany jest trójkąt $ABC$.
	Znaleźć punkt $F$ taki, by suma $|FA| + |FB| + |FC|$ była możliwie najmniejsza.
\end{problem}

Powyższe zadanie rozwiązał Evangelista Torricelli, który dostał je w formie wyzwania od Fermata.
Rozwiązanie opublikował student Torricelliego, Viviani, w 1659 roku.
% TODO: Johnson, R. A. Modern Geometry: An Elementary Treatise on the Geometry of the Triangle and the Circle. Boston, MA: Houghton Mifflin, pp. 221-222, 1929.

% TODO: rozwiązanie https://en.wikipedia.org/wiki/Napoleon%27s_problem

\color{black}



Twierdzenie Brianchona, Newtona, van Aubela.
Wzór Routha, równosć Gergonne'a.
Twierdzenie Steinera.
Czewiany, symediany, punkt Lemoine'a.

Okręgi Apoloniusza, dwustosunek.


\chapter{Przekształcenia płaszczyzny}
Izometrie, punkty stałe.
Translacje, symetrie osiowe, symetrie środkowe, obroty.
Twierdzenie Chasles'a: każda izometria płaszczyzny jest złożeniem co najwyżej trzech symetrii osiowych.
Symetria osiowa z poślizgiem.
Słowo Banacha.
Klasyfikacja podobieństw.
Okrąg siedmiu punktów. % https://mathworld.wolfram.com/BrocardCircle.html ?
Przekształcenia afiniczne i rzutowe.

% https://www.cut-the-knot.org/Curriculum/Geometry/HeronsProblem.shtml
% This one is a basic optimization problem. It's quite famous, being discussed in Heron's Catoptrica (On Mirrors from the Greek word Katoptron Catoptron = Mirror) that, in all likelihood, saw the light of day some 2000 years ago.


\section{Do zrobienia po angielsku}
Point -- punkt.
Line segment, endpoints.
Midpoint, equidistant.
Ray, origin, line.
Collinear / Concurrent.
Plane, planar.

Circle, locus (miejsce geometryczne), radius, tangent, secant, chord, diameter, arc.

Compass and straightedge.
Ruler and protractor.

Angle, vertex, sides.

Acute, obtuse, right.
Perpendicular.
Adjacent, reflex, straight angle.
Supplementary, complementary, vertical angles.
Corresponding, alternate, same-side. Interior, exterior.

Triangle.

Parallel.
Transversal?

\chapter{Gryzmoły}

\section{Geometria I UW (sylabus)}
\begin{enumerate}
	\item -- \begin{enumerate}
		\item Przystawanie figur na płaszczyźnie.
		\item Cechy przystawania trójkątów.
		\item Własności równoległoboków.
		\item Problem Fagnano i problem Fermata.
		\item Kąty w okręgu: wpisane, kąty środkowe i kąty dopisane.
		\item Twierdzenia o kątach wpisanych, kątach środkowych i kątach dopisanych do okręgu.
		\item Kątowe warunki na istnienie okręgu przechodzącego przez cztery punkty.
		\item Zastosowanie: okrąg dziewięciu punktów, twierdzenie o prostej Simsona.
		\item Styczna do okręgu, okrąg wpisany w kąt.
		\item Okrąg wpisany w trójkąt, okręgi dopisane do trójkąta.
		\item Warunki istnienia okręgu stycznego do czterech prostych.
	\end{enumerate}
	\item -- \begin{enumerate}
		\item Stosunek podziału wektora.
		\item Twierdzenie Talesa, twierdzenie odwrotne i jego zastosowania.
		\item Pole.
		\item Pola wybranych figur, twierdzenie Pitagorasa.
		\item Pole zorientowane.
		\item Twierdzenie Newtona: środek okręgu wpisanego w czworokąt i środki przekątnych tego czworokąta są współliniowe.
		\item Twierdzenie Gaussa: środki przekątnych czworokąta zupełnego są współliniowe.
		\item Definicja jednokładności, podobieństwo figur.
		\item Cechy podobieństwa trójkątów.
		\item Stosunek pól figur podobnych.
		\item Iloczynowe warunki istnienia okręgu przechodzącego przez cztery punkty.
		\item Pojęcie potęgi punktu względem okręgu.
		\item Twierdzenie Ptolemeusza.
	\end{enumerate}
	\item -- \begin{enumerate}
		\item Wielkości miarowe w trójkącie: wzór Herona, wzory na promienie okręgów wpisanych, dopisanych.
		\item Twierdzenie o dwusiecznej i okrąg Apoloniusza.
		\item Twierdzenie Cevy (wraz z trygonometryczną wersją), przykłady punktów szczególnych trójkąta: punkt Nagela, punkt Gergonne'a, punkt Lemoine'a.
		\item Punkty izogonalnie sprzężone w trójkącie.
		\item Twierdzenie Menelausa.
	\end{enumerate}
	\item -- \begin{enumerate}
		\item Jednokładność.
		\item Konstrukcja obrazu jednokładnego punktu, okręgu, prostej.
		\item Środek jednokładności dwóch trójkątów.
		\item Środki jednokładności dwóch okręgów.
		\item Prosta Eulera w trójkącie (środek okręgu opisanego, środek ciężkości, ortocentrum).
		\item Zastosowanie: Twierdzenie Pascala.
		\item Twierdzenie Kirkmana: jeśli część wspólna dwóch trójkątów wpisanych w okrąg jest sześciokątem wypukłym, to główne przekątne tego sześciokąta przecinają się w jednym punkcie.
		\item Grupa dylatacji na płaszczyźnie.
		\item Twierdzenia o składaniu jednokładności i przesunięć, twierdzenie o środkach jednokładności trzech okręgów.
	\end{enumerate}
	\item -- \begin{enumerate}
		\item Grupa izometrii na płaszczyźnie.
		\item Konstrukcja obrazu punktu, okręgu, prostej przy translacji, obrocie i symetrii osiowej.
		\item Złożenie dwóch i złożenie trzech symetrii osiowych. 
		\item Twierdzenia o składaniu izometrii. Klasyfikacja izometrii na płaszczyźnie. 
		\item Izometrie parzyste i izometrie nieparzyste. Twierdzenie o redukcji.
		\item Twierdzenie Napoleona: środki ciężkości trójkątów równobocznych zbudowanych na bokach dowolnego trójkąta są wierzchołkami trójkąta równobocznego.
	\end{enumerate}
	\item -- \begin{enumerate}
		\item Grupa podobieństw płaszczyzny.
		\item Podobieństwa spiralne i odbicia dylatacyjne.
		\item Klasyfikacja podobieństw płaszczyzny.
	\end{enumerate}
\end{enumerate}

\section{Geometria II UW}
1. Potęga punktu względem okręgu, oś potęgowa dwóch okręgów, środek potęgowy trzech okręgów, twierdzenie Brianchona, konstrukcja stycznej do okręgu samą linijką, okręgi współpękowe, twierdzenie Gaussa-Bodenmillera, twierdzenie o motylku, formuła Eulera na odległość między środkami okręgu opisanego i wpisanego (dla trójkąta), twierdzenie Ponceleta dla trójkąta.

2. Obrazy inwersyjne okręgów i prostych, konforemność inwersji, okręgi stałe inwersji, okręgi prostopadłe, zmiana odległości przy inwersji, twierdzenie Ptolemeusza, zmiana promienia okręgu przy inwersji, łańcuchy Steinera, formuła Kartezjusza, formuła Fussa dla czworokątów, twierdzenie Feuerbacha.

3. Ogniska elipsy i hiperboli, ognisko, kierownica i mimośród stożkowych, asymptoty hiperboli, konstrukcja stycznej do stożkowej, rzuty ustalonego ogniska na styczne, własności izogonalne stożkowych, równania kanoniczne stożkowych, elipsa jako przekrój walca. Ognisko, kierownica i mimośród stożkowej na przekroju stożka. Przekroje stożków ze sferami wpisanymi. Równanie ogólne stożkowej w układzie współrzędnych, duży i mały wyznacznik. Równania stożkowych we współrzędnych biegunowych.

4. Grupa przekształceń afinicznych od strony geometrycznej: powinowactwa osiowe, rozkład przekształcenia afinicznego na podobieństwo i powinowactwo osiowe, kierunki główne przekształcenia afinicznego, niezmienniczość stosunku pól przy przekształceniu afinicznym, obraz okręgu przy przekształceniu afinicznym
Literatura: 	

1. E. H. Askwith, D.D. A Course of Pure Geometry, Cambridge 1917.

2. H. Fukagawa, D. Pedoe, Japanese temple geometry problems. Sangaku, Charles Babbage Research Centre, Winnipeg 1989.

3. R. A. Johnson, Advanced Euclidean geometry: An elementary treatise on the geometry of the triangle and the circle, Dover Publications, Inc., New York 1960.

4. W. Pompe, Geometria kół, Wydawnictwo Szkolne OMEGA, Kraków 2019.

5. V. Prasolov, Zadaczi po planimietrii. Tom I-II (ros.), Nauka, Moskwa 1991

\section{Geometria III}
Geometria rzutowa: ujęcie od strony geometrycznej. Płaszczyzna rzutowa (rzeczywista), przekształcenia rzutowe prostych, pęków, stożkowych, pęków stycznych do stożkowych. Twierdzenia Desarguesa, Pappusa, Pascala, Brianchona. Dualność: biegun i biegunowa względem okręgu i stożkowych. Sprzężenie biegunowe. Inwolucje rzutowe, twierdzenia inwolucyjne. Pęki okręgów i stożkowych jako generatory inwolucji. Twierdzenie Ponceleta. Stożkowe w ujęciu rzutowym, twierdzenia Steinera i Braikenridge'a-Maclaurina. Rzutowe określenie ogniska i kierownicy stożkowych. Punkty urojone przecięcia prostej ze stożkową w ujęciu czysto geometrycznym.

\raggedright
\indexprologue{\small Tekst prologu...}
\printindex

\indexprologue{\small Tekst prologu...}
\printindex[persons]

\end{document}

\subsection{Podsekcja do zrobienia}
\subsubsection{Podpodsekcja do zrobienia}
Aksjomaty. Kąty naprzemianległe i odpowiadające.
Przystawanie trójkątów.
Pons asionorum.
Łamane i wielokąty.
Równoległobok.
Równoważność wektorów.
Symetria osiowa.
Symetralna.
Styczna do okręgu.
Kąty środkowe i wpisane.
Cykliczność. Prosta Wallace'a.
Ortocentrum i trójkąt ortyczny.
Twierdzenie Miquela.
Dwusieczna. Okrąg wpisany i dopisane.
Twierdzenie Pitagorasa.
Twierdzenie Talesa.
Podobieństwo.
Twierdzenie Ptolemeusza.
Twierdzenie Carnota.
Potęga punktu względem okręgu.
Okrąg Apoloniusza.
Pęki okręgów.
Twierdzenie Eulera.
Twierdzenie Morleya.
Trygonometria. Wzór Herona.
Twierdzenie Urquharta.
Kąt Crelle'a-Brocarda.
Twierdzenie o siódmym okręgu.
Współliniowość.
Współpękowość.
Ceva i Menelaos.
Twierdzenie Ponceleta.
Jednokładność.
Inwersja.
Dwustosunek.
