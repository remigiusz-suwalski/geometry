\subsection{Geometria I}
\subsubsection{X}
1. Przystawanie figur na płaszczyźnie. Cechy przystawania trójkątów. Własności równoległoboków. Problem Fagnano i problem Fermata. Kąty w okręgu: wpisane, kąty środkowe i kąty dopisane. Twierdzenia o kątach wpisanych, kątach środkowych i kątach dopisanych do okręgu. Kątowe warunki na istnienie okręgu przechodzącego przez cztery punkty. Zastosowanie: okrąg dziewięciu punktów, twierdzenie o prostej Simsona. Styczna do okręgu, okrąg wpisany w kąt. Okrąg wpisany w trójkąt, okręgi dopisane do trójkąta. Warunki istnienia okręgu stycznego do czterech prostych.

\subsubsection{X}
2. Stosunek podziału wektora. Twierdzenie Talesa, twierdzenie odwrotne i jego zastosowania. Pole. Pola wybranych figur, twierdzenie Pitagorasa. Pole zorientowane. Twierdzenie Newtona: środek okręgu wpisanego w czworokąt i środki przekątnych tego czworokąta są współliniowe. Twierdzenie Gaussa: środki przekątnych czworokąta zupełnego są współliniowe. Definicja jednokładności, podobieństwo figur. Cechy podobieństwa trójkątów. Stosunek pól figur podobnych. Iloczynowe warunki istnienia okręgu przechodzącego przez cztery punkty. Pojęcie potęgi punktu
względem okręgu. Twierdzenie Ptolemeusza.

\subsubsection{X}
3. Wielkości miarowe w trójkącie: wzór Herona, wzory na promienie okręgów wpisanych, dopisanych. Twierdzenie o dwusiecznej i okrąg Apoloniusza. Twierdzenie Cevy (wraz z trygonometryczną wersją), przykłady punktów szczególnych trójkąta: punkt Nagela, punkt Gergonne'a, punkt Lemoine'a. Punkty izogonalnie sprzężone w trójkącie. Twierdzenie Menelausa.

\subsubsection{Jednokładność}
Jednokładność.

Konstrukcja obrazu jednokładnego punktu, okręgu, prostej.

Środek jednokładności dwóch trójkątów.

Środki jednokładności dwóch okręgów.

Prosta Eulera w trójkącie (środek okręgu opisanego, środek ciężkości, ortocentrum).

Zastosowanie: Twierdzenie Pascala.

Twierdzenie Kirkmana: jeśli część wspólna dwóch trójkątów wpisanych w okrąg jest sześciokątem wypukłym, to główne przekątne tego sześciokąta przecinają się w jednym punkcie.

Grupa dylatacji na płaszczyźnie.

Twierdzenia o składaniu jednokładności i przesunięć, twierdzenie o środkach jednokładności trzech okręgów.

\subsubsection{Izometrie}
5. Grupa izometrii na płaszczyźnie. Konstrukcja obrazu punktu, okręgu, prostej przy translacji, obrocie i symetrii osiowej. Złożenie dwóch i złożenie trzech symetrii osiowych. Twierdzenia o składaniu izometrii. Klasyfikacja izometrii na płaszczyźnie. Izometrie parzyste i izometrie nieparzyste. Twierdzenie o redukcji. Twierdzenie Napoleona: środki ciężkości trójkątów równobocznych zbudowanych na bokach dowolnego trójkąta są wierzchołkami trójkąta równobocznego.

\subsubsection{Podobieństwa}
6. Grupa podobieństw płaszczyzny. Podobieństwa spiralne i odbicia dylatacyjne. Klasyfikacja podobieństw płaszczyzny.
