% strona pierwsza
\thispagestyle{empty}
{\noindent\fontsize{18pt}{18pt}\selectfont Biblioteka Aleksandryjska, tom I}

\noindent\makebox[\linewidth]{\rule{\textwidth}{1pt}}

\newpage

% strona druga
\thispagestyle{empty}
\phantom{nothing}
\newpage

% strona trzecia
\thispagestyle{empty}
{\noindent\fontsize{18pt}{18pt}\selectfont Epafrodyt z Ptolemais}

\noindent\makebox[\linewidth]{\rule{\textwidth}{1pt}}

\vspace{10mm}

{\noindent\fontsize{24pt}{24pt}\selectfont \textbf{Geometria}}
\vspace{10mm}

{\noindent\fontsize{14pt}{14pt}\selectfont Wydanie zerowe (eksperymentalne)}

\newpage

% strona czwarta
\thispagestyle{empty}
\begin{figure}[H]
\begin{minipage}[b]{.48\linewidth}
{\noindent Epafrodyt z Eudoksos\\
do napisania\\
do napisania\\
do napisania}
\end{minipage}
\begin{minipage}[b]{.48\linewidth}
{\noindent do napisania\\
do napisania\\
do napisania\\
do napisania}
\end{minipage}
\end{figure}

{\noindent \textbf{Kategorie MSC 2020}\\do napisania} \vspace{5mm}

{\noindent \textbf{Tytuł oryginału}\\do napisania} \vspace{5mm}

{\noindent \textbf{Z greki tłumaczyła}\\do napisania} \vspace{5mm}

{\noindent \textbf{Okładkę zaprojektował}\\do napisania} \vspace{5mm}

{\noindent \textbf{Zredagował}\\do napisania} \vspace{5mm}

{\noindent \textbf{Zredagowała technicznie}\\do napisania} \vspace{5mm}

{\noindent \textbf{Złożyli i połamali}\\do napisania} \vspace{5mm}

{\noindent \textbf{Korekty dokonali}\\do napisania} \vfill

{\noindent Copyleft © 2024 by Antykwariat Czarnoksięski.
Książka, a żeby było śmieszniej także każda jej część, mogą być przedrukowywane oraz w jakikolwiek inny sposób reprodukowane czy powielane mechanicznie, fotooptycznie, zapisywane elektronicznie lub magnetycznie, oraz odczytywane w środkach publicznego przekazu bez pisemnej zgody wydawcy.
}

\vspace{5mm}
{
    \noindent
    Tekst udostępniany na licencji Creative Commons: uznanie autorstwa, użycie niekomercyjne. Przeczytaj więcej na \texttt{https://creativecommons.org/licenses/by-nc/4.0/deed.pl}.
}

\vspace{5mm}

{\noindent Przygotowano w systemie \TeX, wydrukowano na siarczystym papierze.}

% strona piąta
\newpage
\section*{Przedmowa}
Do napisania.

\begin{flushright}
Epafrodyt,\\gdzie, kiedy
\end{flushright}

\tableofcontents
% \pagestyle{fancy} % Enable default headers and footers again
\cleardoublepage % Start the following content on a new page