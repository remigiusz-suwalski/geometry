\subsection{Aksjomaty Euklidesa}

\begin{euclidelement}[opisanie 1.1]
	Punktem lub znakiem jest, co nie ma żadnych części, lub co nie ma żadnej wielkości.
\end{euclidelement}

\begin{euclidelement}[opisanie 1.11]
	Kąt rozwarty jest ten, który jest większy od kąta prostego.
\end{euclidelement}

\begin{euclidelement}[opisanie 1.21]
	Figury trójkątne prostokreślne są te, które są ograniczone trzema liniami prostymi.
\end{euclidelement}

% TODO: https://kpbc.umk.pl/dlibra/publication/37/edition/66/content

{\color{red}
Euklides wyróżnił kilka pojęć pierwotnych (takich jak punkt, który był dla Euklidesa \emph{tym, co nie ma żadnych części}) i pięć aksjomatów, przytoczonych za książką \emph{O Elementach Euklidesa}:

\begin{enumerate}
	\item Zakłada się, że od każdego punktu do każdego punktu można poprowadzić linię prostą.
	\item I że ograniczoną prostą można ciągle przedłużać po prostej.
	\item I że z każdego środka każdym rozwarciem można zakreślić kolo.
	\item I że wszystkie kąty proste są równe między sobą.
	\item I jeżeli prosta padająca na dwie proste tworzy po jednej stronie kąty wewnętrzne, które w sumie są mniejsze od dwóch prostych, to te proste przedłużone nieograniczenie schodzą się po tej stronie, po której kąty te w sumie są mniejsze od dwóch prostych.
\end{enumerate}

Pojęcia pierwotne i aksjomaty Euklidesa nie są jednak idealne.
Dlatego zamiast nich będziemy używać aksjomatów Hilberta podanych około 1899 roku.
}